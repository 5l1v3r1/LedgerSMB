<?lsmb FILTER latex { format=FORMAT+"(xelatex)" } -?>
\documentclass{scrartcl}
 \usepackage{xltxtra}
 \usepackage{fontspec}
 \setmainfont{LiberationSans-Regular.ttf}
\usepackage{tabularx}
\usepackage{longtable}
\usepackage[letterpaper,top=2cm,bottom=1.5cm,left=1.1cm,right=1.5cm]{geometry}
\usepackage{graphicx}

\begin{document}

\pagestyle{myheadings}
\thispagestyle{empty}

\newsavebox{\ftr}
\sbox{\ftr}{
  \parbox{\textwidth}{
  \tiny
   \rule[1.5em]{\textwidth}{0.5pt}
<?lsmb text('Items returned are subject to a 10% restocking charge.') ?>
<?lsmb text('A return authorization must be obtained from [_1] before goods are returned. Returns must be shipped prepaid and properly insured. [_1] will not be responsible for damages during transit.', company) ?>
  }
}

<?lsmb INCLUDE letterhead ?>

% Breaking old pagebreak directive
%<?xlsmb pagebreak 65 27 37 ?>
%\end{tabularx}
%
%\newpage
%
%\markboth{<?xlsmb company ?>\hfill <?xlsmb ordnumber ?>}{<?xlsmb company ?>\hfill <?xlsmb ordnumber ?>}
%
%\begin{tabularx}{\textwidth}{@{}rlXllrrl@{}}
%  \textbf{Item} & \textbf{Number} & \textbf{Description} & \textbf{Serial Number} & & \textbf{Qty} & \textbf{Ship} & \\
%<?xlsmb end pagebreak ?>


\vspace*{0.5cm}

\parbox[t]{.5\textwidth}{
\textbf{Ship To}} \hfill

\vspace{0.3cm}

\parbox[t]{.5\textwidth}{

<?lsmb shiptoname ?>

<?lsmb shiptoaddress1 ?>

<?lsmb shiptoaddress2 ?>

<?lsmb shiptocity ?>
<?lsmb IF shiptostate ?>
\hspace{-0.1cm}, <?lsmb shiptostate ?>
<?lsmb END ?>
<?lsmb shiptozipcode ?>

<?lsmb shiptocountry ?>
}
\parbox[t]{.5\textwidth}{
  <?lsmb shiptocontact ?>

  <?lsmb IF shiptophone ?>
  Tel: <?lsmb shiptophone ?>
  <?lsmb END ?>

  <?lsmb IF shiptofax ?>
  Fax: <?lsmb shiptofax ?>
  <?lsmb END ?>

  <?lsmb shiptoemail ?>
}
\hfill

\vspace{1cm}

\textbf{\MakeUppercase{<?lsmb text('Packing List') ?>}}
\hfill

\vspace{1cm}

\begin{tabularx}{\textwidth}{*{7}{|X}|} \hline
  \textbf{<?lsmb text('Invoice #') ?>} & \textbf{<?lsmb text('Order #') ?>}
  & \textbf{<?lsmb text('Date') ?>} & \textbf{<?lsmb text('Contact') ?>}
  <?lsmb IF warehouse ?>
  & \textbf{<?lsmb text('Warehouse') ?>}
  <?lsmb END ?>
  & \textbf{<?lsmb text('Shipping Point') ?>}
  & \textbf{<?lsmb text('Ship via') ?>} \\ [0.5em]
  \hline

  <?lsmb invnumber ?> & <?lsmb ordnumber ?>
  <?lsmb IF shippingdate ?>
  & <?lsmb shippingdate ?>
  <?lsmb ELSE ?>
  & <?lsmb transdate ?>
  <?lsmb END ?>
  & <?lsmb employee ?>
  <?lsmb IF warehouse ?>
  & <?lsmb warehouse ?>
  <?lsmb END ?>
  & <?lsmb shippingpoint ?> & <?lsmb shipvia ?> \\
  \hline
\end{tabularx}

\vspace{1cm}

\begin{longtable}{@{\extracolsep{\fill}}rllllrrl@{}}
  \textbf{<?lsmb text('Item') ?>} & \textbf{<?lsmb text('Number') ?>}
  & \textbf{<?lsmb text('Description') ?>}
  & \textbf{<?lsmb text('Serial Number') ?>} &
  & \textbf{<?lsmb text('Qty') ?>} & \textbf{<?lsmb text('Ship') ?>} & \\

<?lsmb FOREACH number ?>
<?lsmb lc = loop.count - 1 ?>
  <?lsmb runningnumber.${lc} ?> &
  <?lsmb number.${lc} ?> &
  <?lsmb item_description.${lc} ?> &
  <?lsmb serialnumber.${lc} ?> &
  <?lsmb deliverydate.${lc} ?> &
  <?lsmb qty.${lc} ?> &
  <?lsmb ship.${lc} ?> &
  <?lsmb unit.${lc} ?> \\
<?lsmb END ?>
\end{longtable}


\parbox{\textwidth}{
\rule{\textwidth}{2pt}

\vspace{12pt}

<?lsmb FOREACH P IN notes.split('\n\n') ?>
<?lsmb P ?>\medskip

<?lsmb END ?>

}

\vfill

\rule{\textwidth}{0.5pt}

\usebox{\ftr}

\end{document}
<?lsmb END ?>
