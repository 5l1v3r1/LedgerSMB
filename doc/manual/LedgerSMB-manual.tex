%% LyX 1.4.2 created this file.  For more info, see http://www.lyx.org/.
%% Do not edit unless you really know what you are doing.
\documentclass[english]{article}
\usepackage[T1]{fontenc}
\usepackage[latin1]{inputenc}
\IfFileExists{url.sty}{\usepackage{url}}
                      {\newcommand{\url}{\texttt}}

\newcommand{\version}{1.2}
\makeatletter

%%%%%%%%%%%%%%%%%%%%%%%%%%%%%% LyX specific LaTeX commands.
\providecommand{\LyX}{L\kern-.1667em\lower.25em\hbox{Y}\kern-.125emX\@}
%% Bold symbol macro for standard LaTeX users
\providecommand{\boldsymbol}[1]{\mbox{\boldmath $#1$}}

%% Because html converters don't know tabularnewline
\providecommand{\tabularnewline}{\\}

%%%%%%%%%%%%%%%%%%%%%%%%%%%%%% User specified LaTeX commands.

\usepackage{metatron}

\renewcommand{\abstractname}{Executive Summary}
\title{Ledger-SMB Manual v. \version}
\author{The LedgerSMB Core Team}
\date{\today}


\usepackage{babel}
\makeatother
\begin{document}
\maketitle
Copyright \copyright 2006 The LedgerSMB Core Team. Permission
is granted to copy, distribute and/or modify this document under the
terms of the GNU Free Documentation License, Version 1.2 or any later
version published by the Free Software Foundation; with no Invariant
Sections, no Front-Cover Texts, and no Back-Cover Texts. A copy of
the license is included in the section entitled \char`\"{}GNU Free
Documentation License\char`\"{} (Appendix \ref{fdl}).

\tableofcontents{}

\listoffigures


\clearpage


\part{Ledger-SMB and Business Processes}


\section{Introduction to Ledger-SMB}


\subsection{Why Ledger-SMB}


\subsubsection{Advantages of Ledger-SMB}

\begin{itemize}
\item Flexibility and Central Management 
\item Accessibility over the Internet (for some users) 
\item Relatively open data format
\item Integration with other tools 
\item Excellent accounting options for Linux users
\item Open Source 
\item Flexible, open framework that can be extended or modified to fit
your business. 
\item Security-conscious development community. 
\end{itemize}

\subsubsection{Key Features}

\begin{itemize}
\item Accounts Receivable 

\begin{itemize}
\item Track sales by customer 
\item Issue Invoices, Statements, Receipts, and more 
\item Do job costing and time entry for customer projects 
\item Manage sales orders and quotations 
\item Ship items from sales orders 
\end{itemize}
\item Accounts Payable 

\begin{itemize}
\item Track purchases and debts by vendor 
\item Issue RFQ's Purchase Orders, etc. 
\item Track items received from purchase orders 
\end{itemize}
\item Budgeting 

\begin{itemize}
\item Track expenditures and income across multiple departments 
\item Track all transactions across departments 
\end{itemize}
\item Check Printing 

\begin{itemize}
\item Customize template for any check form 
\end{itemize}
\item General Ledger 
\item Inventory Management 

\begin{itemize}
\item Track sales and orders of parts 
\item Track cost of goods sold using First In/First Out method 
\item List all parts below reorder point 
\item Track ordering requirements 
\item Track, ship, receive, and transfer parts to and from multiple warehouses 
\end{itemize}
\item Localization 

\begin{itemize}
\item Provide Localized Translations for Part Descriptions 
\item Provide Localized Templates for Invoices, Orders, Checks, and more 
\item Select language per customer, invoice, order, etc. 
\end{itemize}
\item Manufacturing 

\begin{itemize}
\item Track cost of goods sold for manufactured goods (assemblies) 
\item Create assemblies and stock assemblies, tracking materials on hand 
\end{itemize}
\item Multi-company/Multiuser 

\begin{itemize}
\item One isolated database per company 
\item Users can have localized systems independent of company data set 
\end{itemize}
\item Point of Sale 

\begin{itemize}
\item Run multiple cash registers against main Ledger-SMB installation 
\item Suitable for retail stores and more
\item Credit card processing via TrustCommerce 
\item Supports some POS hardware out of the box including:

\begin{itemize}
\item Logic Controls PD3000 pole displays (serial or parallel)
\item Basic text-based receipt printers
\item Keyboard wedge barcode scanners
\item Keyboard wedge magnetic card readers
\item Printer-attached cash drawers
\end{itemize}
\end{itemize}
\item Price Matrix 

\begin{itemize}
\item Track different prices for vendors and customers across the board 
\item Provide discounts to groups of customers per item or across the board 
\item Store vendors' prices independent of the other last cost in the parts
record 
\end{itemize}
\item Reporting 

\begin{itemize}
\item Supports all basic financial statements
\item Easily display customer history, sales data, and additional information 
\item Open framework allows for ODBC connections to be used to generate
reports using third party reporting tools. 
\end{itemize}
\item Tax 

\begin{itemize}
\item Supports Retail Sales Tax and Value Added Tax type systems 
\item Flexible framework allows one to customize reports to change the tax
reporting framework to meet any local requirement. 
\end{itemize}
\end{itemize}

\subsection{Limitations of Ledger-SMB}

\begin{itemize}
\item No payroll module (Payroll must be done manually) 
\item Some integration limitations 
\item Further development/maintenance requires a knowledge of a relatively
broad range of technologies 
\end{itemize}

\subsection{System Requirements of Ledger-SMB}

\begin{itemize}
\item PostgreSQL 
\item A CGI-enabled Web Server (for example, Apache) 
\item Perl 5.8.x
\item An operating system which supports the above software (usually Linux,
though Windows, MacOS X, etc. do work) 
\item \LaTeX{}\ (optional) is required to create PDF or Postscript invoices 
\item The following CPAN modules:

\begin{itemize}
\item Data::Dumper
\item Locale::Maketext
\item Locale::Maketext::Lexicon
\item MIME::Base64
\item Digest::MD5
\item HTML::Entities
\item DBI
\item DBD::Pg
\item Math::BigFloat
\item IO::File
\item Encode
\item Locale::Country
\item Locale::Language
\item Time::Local
\item Cwd
\item Config::Std
\item MIME::Lite
\end{itemize}
\end{itemize}

\section{User Account and Database Administration Basics}

These functions are accessed via the admin.pl script in the
installed directory of Ledger-SMB.


\subsection{Companies and Datasets}

Ledger-SMB stores its information in locale-specific data sets. When
a dataset is created, it sets various defaults such as currency, a
basic chart of accounts setup, and so forth. Note that the default
setup is for Canada, where the author of the software resides.

Datasets are stored as PostgreSQL databases. The application
is designed with the idea that each dataset will represent exactly
one company. If a customer is working with multiple companies, he/she
must create a dataset to for each.

When creating a dataset, the application asks for both a username and a 
superusername.  If the superuser's information is not filled in, Ledger-SMB will
attempt to populate an existing dataset, but if this information is filled in,
the program will log into the PostgreSQL cluster with the superusername and
password, create the database, and attempt to add Plpgsql to it.

\subsection{How to Create a User}

Users are created by going to the admin.pl page and clicking on \char`\"{}Add
User.\char`\"{} One then fills out the form and when it is saved,
the user is created.


\subsection{Permissions}

The permissions system is not rigorously enforced within Ledger-SMB,
in the sense that the permissions API is generally not used in the
application itself. Instead permissions are used to enable/disable
menu options. Setting an enforcement of such permissions would require
some custom programming at the present time. Most organizations, however,
find that the current system is adequate.

The checkboxes which are marked enable menu entries. Those that are
unchecked disable those entries on the menu.


\subsection{User Account Types}

\begin{itemize}
\item User is a general user of the system 
\item Managers often are able to see a larger amount of data 
\item Administrators have full access to the system 
\end{itemize}

\subsection{Other Features}

\begin{itemize}
\item Lock System allows one to lock users out of the system while maintenance
is performed. This is only necessary during upgrades or maintenance
which results in the RDBMS being offline. 
\item Change Admin Password changes the administrative password.
\item Logout terminates the admin session. 
\end{itemize}

\section{Chart of Accounts}

The Chart of Accounts provides a basic overview of the logical structure
of the accounting program. One can customize this chart to allow for
tracking of different sorts of information.


\subsection{Introduction to Double Entry Bookkeeping}
In order to set up your chart of accounts in LedgerSMB you will need to
understand a bit about double entry bookkeeping.  This section provides a
brief overview of the essential concepts.  There is a list of references
for further reading at the end.

\subsubsection{Business Entity}
You don't want to mix your personal expenses and income with that of the
business or you will not be able to tell how much money it is making (if
any).  For the same reason you will want to keep track of how much money
you put into and take out of the business so you will want to set up a
completely seperate set of records for it and treat it almost as if it had
a life of its own.

\subsubsection{Double Entry}
Examples:
\begin{itemize}
\item When you buy you pay money and receive goods.
\item When you sell you get money and give goods.
\item When you borrow you get money and give a promise to pay it back.
\item When you lend you give money and get a promise to pay it back.
\item When you sell on credit you give goods and get a promise to pay.
\item When you buy on credit you give a promise to pay and get goods.
\end{itemize}

You need to record both sides of each transaction: thus double entry.
Furthermore, you want to organize your entries, recording those having to
do with money in one place, value of goods bought and sold in another,
money owed in yet another, etc.  Hence you create accounts, and record each
half of each transaction in an appropriate account.  Of course, you won't
have to actually record the amount in more than one place yourself: the
program takes care of that.

\subsubsection{Accounts}

\begin{description}
\item[Assets] Valuable stuff the business owns such as money and
goods available for sale 
\item[Liabilities] Debts owned by the business such as bank loans and unpaid 
bills
\item[Equity or Capital] What would be left for the owner if all the assets were 
converted to money and all the liabilities paid off ("Share Capital" on the
LedgerSMB default chart of accounts: not to be confused with "Capital Assets".)
\item[Revenue] Income from business activity: increases Equity  
\item[Expense] The light bill, the cost of goods sold, etc: decreases Equity
\end{description}

All other accounts are subdivisions of these.  The relationship between the
top-level accounts is often stated in the form of the Accounting Equation
(don't worry: you won't have to solve it):

Assets = Liabilities + Equity + (Revenue - Expense)

You won't actually use this equation while doing your bookkeeping, but it's
a useful tool for understanding how the system works.

\subsubsection{Debits and Credits}

Traditional paper accounting systems used a two-column form in which all
increases went in one column and all deceases in the other.  For asset and
expense accounts increases went in the left column and decreases in the
right.  For liability and capital accounts decreases went in the left
column and increases in the right.  Looking at the accounting equation we
see that assets are on the left, so it is logical that asset increases
would go on the left.  Libilities, capital, and revenue are on the right so
it is logical that their increase would go on the right.  Expenses,
however, are on the right, so why do their increases go on the left?
Because expenses are subtracted from the right side of the equation and so
expense increases decrease the right side of the equation.

Entries in the left column of the traditional form are called debits, while
entries on the right are called credits.  Neither is "negative".

\begin{itemize}
\item Debits increase assets
\item Debits increase expense
\item Credits increase liabilities
\item Credits increase capital
\item Credits increase revenue
\end{itemize}

Examples:

You go to the bank and make a deposit.  The teller tells you that he is
going to credit your account.  This is correct: your account is money the
bank owes you and so is a liability from their point of view.  Your deposit
increased this liability and so they will credit it.  They will make an
equal debit to their cash account.  When you return you will debit your
bank deposits account because you have increased that asset and credit cash
on hand because you have decreased that one.


\subsubsection{Accrual}

Early accounting systems were usually run on a cash basis.  One generally did 
not consider money owed to affect the financial health of a company, so expenses
posted when paid as did income.

The problem with this approach is that it becomes very difficult or impossible 
to truly understand the exact nature of the financial health of a business.  One
cannot get the full picture of the financial health of a business because 
outstanding debts are not considered.  Futhermore, this does not allow for 
revenue to be tied to cost effectively, so it becomes difficult to assess how
profitable a given activity truly is.

To solve this problem, accrual-based systems were designed.  The basic principle
is that income and expense should be posted as they are incurred, or accrued.  
This allows one to track income relative to expense for specific projects or 
operations, and make better decisions about which activities will help one 
maximize profitability.

To show how these systems differ, imagine that you bill a customer for time and 
materials for a project you have just completed.  The customer pays the bill 
after 30 days.  In a cash based system, you would post the income at the time 
when the customer pays, while in an accrual system, the income is posted at the
time when the project is completed.

\subsubsection{References}

\url{http://www.accounting-and-bookkeeping-tips.com/learning-accounting/accounting-basics-credit.htm}\\
Discussion of debits and credits as well as links to other accounting subjects.\\

\noindent \url{http://www.computer-consulting.com/accttips.htm}\\
Discussion of double entry bookkeeping.\\

\noindent \url{http://www.minnesota.com/~tom/sql-ledger/howtos/}\\
A short glossary, some links, and a FAQ (which makes the "credit=negative
number" error).  The FAQ focuses on SQL-Ledger, LedgerSMB's ancestor.\\

\noindent \url{http://bitscafe.com/pub2/etp/sql-ledger-notes\#expenses}\\
Some notes on using SQL-Ledger (LedgerSMB's ancestor).\\

\noindent \url{http://en.wikipedia.org/wiki/List\_of\_accounting\_topics}\\
Wikipedia articles on accounting.\\

\noindent \url{http://www.bized.ac.uk/learn/accounting/financial/index.htm}\\
Basic accounting tutorial.\\

\noindent \url{http://www.asset-analysis.com/glossary/glo\_index.html}\\
Financial dictionary and glossary.\\

\noindent \url{http://www.geocities.com/chapleaucree/educational/FinanceHandbook.html}\\
Financial glossary.\\

\noindent \url{http://www.quickmba.com/accounting/fin/}\\
Explanation of fundamentals of accounting, including good discussions
of debits and credits and of double-entry.


\subsection{General Guidelines on Numbering Accounts}

In general, most drop-down boxes in Ledger-SMB order the accounts
by account number. Therefore by setting appropriate account numbers,
one can affect the default values.

A second consideration is to try to keep things under each heading
appropriate to that heading. Thus setting an account number for a
bank loan account in the assets category is not generally advisable.


\subsection{Adding/Modifying Accounts}

These features are listed under System-\textgreater Chart of Accounts.
One can list the accounts and click on the account number to modify
them or click on the \char`\"{}add account\char`\"{} option to create
new accounts.

\begin{itemize}
\item Headings are just broad categories and do not store values themselves,
while accounts are used to store the transactional information. 
\item One cannot have an account that is  a summary account (like AR)
and also has another function. 
\item GIFI is mostly of interest to Canadian customers but it can be used
to create reports of account hierarchies. 
\end{itemize}

\subsection{Listing Account Balances and Transactions}

One can list the account balances via the Reports-\textgreater Chart
of Accounts report. Clicking on the account number will provide a
ledger for that account.


\section{Administration}

This section covers other (non-Chart of Accounts) aspects to the
setup of the Ledger-SMB accounting package. These are generally accessed
in the System submenu.


\subsection{Taxes, Defaults, and Preferences}
One of the new features in 1.2 is the modular sales tax system and the simple
sales tax module.  This allows one to install different tax modules and then 
select which taxes are applied by which programming modules.  The sales tax 
module has access to everything on the submitted form so it is able to make
complex determinations on what is taxable based on arbitrary criteria.

The tax rules drop-down box allows one to select any installed tax module 
(LedgerSMB 1.2 ships only with the simple module), while the ordering is an
integer which allows one to specify a tax run which occurs on the form after
any rules with lower entries in this box.  This allows for compounding of sales tax (for example, when PST applies to the total and GST as well).

\subsubsection{Adding A Sales Tax Account}

Sales Tax is collected on behalf of a state or national government
by the individual store. Thus a sales tax account is a liability--
it represents money owed by the business to the government.

To add a sales tax account, create an account in the Chart of Accounts
as a liability account, check all of the \char`\"{}tax\char`\"{} checkboxes,
and answer the following question as \char`\"{}yes:\char`\"{}

\char`\"{}Include this account on the customer/vendor forms to flag
customer/vendor as taxable?\char`\"{}

Once this account is created, one can set the tax amount.


\subsubsection{Setting a Sales Tax Amount}

Go to System-\textgreater Defaults and the tax account will be listed
near the bottom of the page. The rate can be set there.


\subsubsection{Default Account Setup}

These accounts are the default accounts for part creation and foreign
exchange tracking.


\subsubsection{Currency Setup}

The US accounts list this as USD:CAD:EUR. One can add other currencies
in here, such as IDR (Indonesian Rupiah), etc. Currencies are separated
by colons.


\subsubsection{Sequence Settings}

These sequences are used to generate user identifiers for quotations,
invoices, and the like. If an identifier is not added, the next number
will be used.

A common application is to set invoices, etc. to start at 1000 in
order to hide the number of issued invoices from a customer.

Leading zeros are preserved.  Other special values which can be embedded using
$<$?lsmb ?$>$ tags include:

\begin{description}
\item[DATE] expands to the current date
\item[YYMMDD] expands to a six-digit version of the date.  The components of 
this date can be re-arranged in any order, so MMDDYY, DDMMYY, 
or even just MMYY are all options.
\item[NAME] expands to the name of the customer or vendor
\item[BUSINESS] expands to the type of business assigned to the customer or 
ventor.
\item[DESCRIPTION] expands to the description of the part.  Valid only for parts.
\item[ITEM] expands to the item field.  Valid only for parts.
\item[PERISCOPE] expands to the partsgroup.  Valid only for parts.
\item[PHONE] expands to the telephoen number for customers and vendors.
\end{description}

\subsection{Audit Control}

Auditability is a core concern of the architects of any accounting
system. Such ensures that any modification to the accounting information
leaves a trail which can be followed to determine the nature of the
change. Audits can help ensure that the data in the accounting system
is meaningful and accurate, and that no foul play (such as embezzlement)
is occurring.


\subsubsection{Explaining transaction reversal}

In paper accounting systems, it was necessary to have a means to authoritatively
track corrections of mistakes. The means by which this was done was
known as \char`\"{}transaction reversal.\char`\"{}

When a mistake would be made, one would then reverse the transaction
and then enter it in correctly. For example, let us say that an office
was renting space for \$300 per month. Let us say that they inadvertently
entered it in as a \$200 expense.

The original transaction would be:

\begin{tabular}{l|r|r}
Account &
Debit &
Credit \tabularnewline
\hline 
5760 Rent &
\$200 &
\tabularnewline
2100 Accounts Payable &
&
\$200\tabularnewline
\end{tabular}

The reversal would be:

\begin{tabular}{l|r|r}
Account &
Debit &
Credit \tabularnewline
\hline 
5760 Rent &
&
\$200\tabularnewline
2100 Accounts Payable &
\$200 &
\tabularnewline
\end{tabular}

This would be followed by re-entering the rent data with the correct
numbers. This was meant to ensure that one did not erase data from
the accounting books (and as such that erasing data would be a sign
of foul play).

Ledger-SMB has a capability to require such reversals if the business
deems this to be necessary. When this option is enabled, existing
transactions cannot be modified and one will need to post reversing
transactions to void existing transactions before posting corrected
ones.

Most accountants prefer this means to other audit trails because it
is well proven and understood by them.


\subsubsection{Close books option}

You cannot alter a transaction that was entered before the closing date.  


\subsubsection{Audit Trails}

This option stores additional information in the database to help
auditors trace individual transactions. The information stored, however,
is limited and it is intended to be supplemental to other auditing
facilities.

The information added includes which table stored the record, which
employee entered the information, which form was used, and what the
action was. No direct financial information is included.


\subsection{Departments}

Departments are logical divisions of a business. They allow for budgets
to be prepared for the individual department as well as the business
as a whole. This allows larger businesses to use Ledger-SMB to meet
their needs.


\subsubsection{Cost v Profit Centers.}

In general business units are divided into cost and profit centers.
Cost centers are generally regarded as business units where the business
expects to lose money and profit centers are where they expect to
gain money. For example, the legal department in most companies is
a cost center.

One of the serious misunderstandings people run up against is that
Ledger-SMB tends to more narrowly define cost and profit centers than
most businesses do. In Ledger-SMB a cost center is any department
of the business that does not issue AR transactions. Although many
businesses may have cost centers (like technical support) where customer
fees may subsidize the cost of providing the service, in Ledger-SMB,
these are profit centers.

Ledger-SMB will not allow cost centers to be associated with AR transactions.
So if you want this functionality, you must create the department
as a profit center.


\subsection{Warehouses}

Ledger-SMB has the ability to track inventory by warehouse. Inventory
items can be moved between warehouses, and shipped from any warehouse
where the item is in stock. We will explore this concept more later.


\subsection{Languages}

Languages allow for goods and services to be translated so that one
can maintain offices in different countries and allow for different
goods and service descriptions to be translated to different languages
for localization purposes.


\subsection{Types of Businesses}

One can create types of businesses and then give them discounts across
the board. For example, one might give a firm that uses one's services
as a subcontractor a 10\% discount or more.


\subsection{Misc.}


\subsubsection{GIFI}

GIFI is a requirement for Canadian customers. This feature allows
one to link accounts with Canadian tax codes to simplify the reporting
process. Some European countries now use a similar system.

People that don't otherwise have a use for GIFI can use it to create reports
which agregate accounts together.

\subsubsection{SIC}

Standard Industrial Classification is a way of tracking the type of
business that a vendor or customer is in. For example, an accountant
would have an SIC of 8721 while a graphic design firm would have an
SIC of 7336. The classification is hierarchical so one could use this
field for custom reporting and marketing purposes.


\subsubsection{Overview of Template Editing}

The templates for invoices, orders, and the like can be edited from
within Ledger-SMB. The submenus within the System submenu such as
HTML Templates, Text Templates and \LaTeX{} templates provide access
to this functionality.


\subsubsection{Year-end}

Although the Year-end functionality in Ledger-SMB is very useful,
it does not entirely make the process simple and painless. One must
still manually enter adjustments prior to closing the books. The extent
to which these adjustments are necessary for any given business is
a matter best discussed with an accountant.

The standard way books are normally closed at the end of the year
is by moving all adjusted%
\footnote{Adjustments would be entered via the General Ledger. The exact process
is beyond the scope of this document, however.%
} income and expenses to an equity account usually called \char`\"{}Retained
Earnings.\char`\"{} Assets and liabilities are not moved. Equity drawing/dividend
accounts are also moved, but the investment accounts are not. The
reasoning behind this process is that one wants a permanent record
of the amount invested in a business, but any dividends ought not
to count against their recipients when new investors are brought on
board.

Ledger-SMB automatically moves all income and expense into the specified
year-end/retained earnings account. It does not move the drawing account,
and this must be done manually, nor does it automate the process of
making adjustments.

Contrary to its name, this function can close the books at any time,
though this would likely be of limited use.


\subsection{Options in the ledger-smb.conf}

The ledger-smb.conf configures the software by assigning site-wide
variables. Most of these should be left alone unless one knows what
one is doing. However, on some systems some options might need to
be changed, so all options are presented here for reference:

\begin{description}
\item[templates] is the directory where the templates are stored. 
\item[sendmail] is the command to use to send a message. It must read the
email from standard input. 
\item[language] allows one to set the language for the login screen and
admin page. 
\item[latex] tells Ledger-SMB whether \LaTeX{} is installed. \LaTeX{} is
required for generating Postscript and PDF invoices and the like. 
\item[Environmental variables] can be set here too. One
can add paths for searching for \LaTeX{}, etc. 
\item[Printers] section can be used to set a hash table of printers for the software.
The primary example is\\
$[$printers$]$\\
Default = lpr\\
Color = lpr -PEpson \\%
 However, this can use any program that can accept print documents
(in Postscript) from standard input, so there are many more possibilities. 
\end{description}


\section{Goods and Services}

The Goods and Services module will focus on the definition of goods
and services and the related accounting concepts.


\subsection{Basic Terms}

\begin{description}
\item [{COGS}] is Cost of Goods Sold. When an item is sold, then the expense
of its purchase is accrued as attached to the income of the sale.
\item [{List}] Price is the recommended retail price. 
\item [{Markup}] is the percentage increase that is applied to the last
cost to get the sell price. 
\item [{ROP}] is re-order point. Items with fewer in stock than this will
show up on short reports. 
\item [{Sell}] Price is the price at which the item is sold. 
\end{description}

\subsection{The Price Matrix}

It is possible to set different prices for different groups of customers,
or for different customers individually. Similarly, one can track
different prices from different vendors along with the required lead
time for an order.


\subsection{Pricegroups}

Pricegroups are used to help determine the discount a given customer
may have.


\subsection{Groups}

Groups represent a way of categorizing POS items for a touchscreen
environment. It is not fully functional yet, but is sufficient that
with some stylesheet changes, it could be made to work.


\subsection{Labor/Overhead}

Labor/overhead is usually used for tracking manufacturing expenses.
It is not directly billed to a customer. It is associated with an
expense/Cost of Goods Sold (COGS) account.


\subsection{Services}

Services include any labor that is billed directly to the customer.
It is associated with an expense/COGS account and an income account.
Services can be associated with sales tax.


\subsubsection{Shipping and Handling as a Service}

One approach to dealing with shipping and handling is to add it as
a service. Create a service called {}``Shipping and Handling'',
with a sell price \$1 per unit, and a 0\% markup. Bill it as \$1 per
unit. This allows one to add the exact amount of shipping and handling
as necessary.


\subsection{Parts}

A part is any single item you might purchase and either might resell
or use in manufacturing an assembly. It is linked to an expense/COGS
account, an income account, and an inventory account. Parts can be
associated with sales tax.


\subsection{Assemblies and Manufacturing}

Manufacturers order parts but they sell the products of their efforts.
Ledger-SMB supports manufacturing using the concept of assemblies.
An assembly is any product which is manufactured on site. It consists
of a selection of parts, services, and/or labor and overhead. Assemblies
are treated as parts in most other regards.

However, one cannot order assemblies from vendors. One must instead
order the components and stock them once they are manufactured.


\subsubsection{Stocking Assemblies}

One stocks assemblies in the Stock Assembly entry on the Goods and
Services submenu. When an assembly is stocked the inventory is adjusted
properly.

The Check Inventory option will cause Ledger-SMB to refuse to stock
an assembly if the inventory required to produce the assembly would
drop the part below the reorder point.


\subsection{Reporting}


\subsubsection{All Items and Parts Reports}

The All Items report provides a unified view of assemblies, parts, services,
and labor for the company, while the Parts report confines it to parts.

Types of reports are: 

\begin{description}
\item [{Active}] lists all items not marked as obsolete. 
\item [{On}] Hand lists current inventory .
\item [{Short}] Lists all items which are stocked below their ROP.
\item [{Obsolete}] Lists all items which are marked as obsolete.
\item [{Orphaned}] Lists all items which have never had a transaction associated
with them. 
\end{description}
One can also list these goods by invoice, order, or quotation.

For best results, it is a good idea to enter some AR and AP data before
running these reports.


\subsubsection{Requirements}

This report is designed to assist managers determine the quantities
of goods to order and/or stock. It compares the quantity on hand with
the activity in a given time frame and provides a list of goods which
need to be ordered and the relevant quantity.


\subsubsection{Services and Labor}

This is similar to the Parts and All Items menu but only supports
Active, Obsolete, and Orphaned reports.


\subsubsection{Assemblies}

This is similar to the Parts and All Items reports but it also provides
an ability to list individual items in the assemblies as well.

AP Invoices, Purchase Orders, and RFQ's are not available on this
report.


\subsubsection{Groups and Pricegroups}

These reports provide a simple interface for locating groups and pricegroups.
The report types are similar to what they are for services.


\subsection{Translations}

One can add translations so that they show up in the customer's native
language in the issued invoice.

To issue translations, one must have languages defined. One can then
add translations to descriptions and part groups.


\subsection{How Cost of Goods Sold is tracked}

Cost of Goods Sold is tracked on a First-In, First-out (FIFO) basis.
When a part is purchased, its cost is recorded in the database. The
cost of the item is then added to the inventory asset account. When
the good is sold, the cost of the item is moved to the cost of goods
sold account.

This means that one must actually provide invoices for all goods entered
at their actual cost. If one enters in \$0 for the cost, the cost
of goods sold will also be \$0 when the item is sold. We will cover
this entire process in more depth after we cover the AP and AR units
below.


\section{AP}


\subsection{Basic AP Concepts}

The Accounts Payable module tracks all financial commitments that
the company makes to other businesses. This includes rent, utilities,
etc. as well as orders of goods and services.


\subsection{Vendors}

A vendor is any business that the company agrees to pay money to.

One can enter vendor information under AP-\textgreater Vendors-\textgreater
Add Vendor. The vendor list can be searched under AP-\textgreater
Vendors-\textgreater Reports-\textgreater Search.

Enter start and end-dates for each vendor. This will make searches
and drop-down boxes faster when some vendors are no longer actively
supporting your company.

A few fields that need explanation are: 

\begin{description}
\item [{BIC}] Bank Identifier Code is often the same as the S.W.I.F.T.
code. This is a code for the bank a customer uses for automated money
transfers. 
\item [{IBAN}] International Bank Account Number is related to the BIC
and is used for cross-border automated money transfers. 
\item [{Terms}] is the number of days one has to pay the invoice. 
\item [{Vendor}] Number is automatically generated. 
\end{description}

\subsection{AP Transactions}

AP Transactions are generally used for items other than goods and
services. Utilities, rent, travel expenses, etc. could be entered
in as an AP transaction.

If the item is paid partially or in full when the transaction is entered,
one can add payments to the payment section.

All other payments can and should be entered under cash payment (below).

The PO Number and Order Number fields are generally used to track
associations with purchase orders sent to vendors, etc. These fields
can be helpful for adding misc. expenses to orders for reporting purposes.

The department drop-down box appears when one has created one or more
departments. A transaction is not required to be associated with a
department, but one can use this feature for budget tracking.

With AP Transactions, there is no option for internal notes. All notes
will appear on any printed version of the transaction.

Note: Printing a transaction does not post it. No data is committed
until the invoice is posted.


\subsection{AP Invoices}

AP Invoices are used to enter in the receipt of goods and services.
Goods and services are deemed entered into the inventory when they
are invoiced.

This screen is reasonably similar to the AP Transaction Screen, though
the part entry section is a bit different.

The AP Invoice section has a capacity to separate internal notes from
notes printed on the invoice. Note, however, that since these are
received invoices, it is rare that one needs this ability.

Note that Ledger-SMB can search for partial part numbers or descriptions.

Also if you have a group you can use this to select the part.

To remove a line item from an invoice or order, delete the partnumber
and click update.


\subsubsection{Correcting an AP Invoice}

If an invoice is entered improperly, the methods used to correct it
will vary depending on whether transaction reversal is enforced or
not. If transaction reversal is not enforced, one can simply correct
the invoice or transaction and repost. Note, however, that this violates 
generally accepted accounting principles.

If transaction reversal is in effect, one needs to create a duplicate 
invoice with exactly opposite values entered. If one part was listed as 
received, then one should enter a negative one for the quantity. Then one 
can enter the invoice number as the same as the old one. Add an R to the 
end to show that it is a reversing transaction. Once this is posted, one can 
enter the invoice correctly.


\subsection{Cash payment And Check Printing}

It is a bad idea to repost invoices/transactions just to enter a payment. 
The Cash-\textgreater Payment window allows one to enter payments against 
AP invoices or transactions.

The printing capability can be used to print checks. The default template
is NEBS 9085, though you can use 9082 as well (as Quickbooks does).

The source field is used to store an identifying number of the source
document, such as the check number. One must select the item to have
it paid, and then enter the amount. One can then print a check.


\subsubsection{Rapid Payment Entry Screen}

One can also use the rapid payment entry screen to print multiple
checks. However, this does not allow you to print the multiple checks
to the screen as a separate document is created for each check. In
this event, one must print directly to a printer as postscript.


\subsection{Transaction/Invoice Reporting}


\subsubsection{Transactions Report}

This report is designed to help you locate AP transactions based on
various criteria. One can search by vendor, invoice number, department,
and the like. One can even search by the shipping method.

The summary button will show what was placed where, while the details
button will show all debits and credits associated with the transaction.

To view the invoice, click on the invoice number. In the detail view,
to view the account transactions as a whole, click on the account
number.

Open invoices are ones not fully paid off, while closed invoices
are those that have been paid.


\subsubsection{Outstanding Report}

The outstanding report is designed to help you locate AP transactions
that are not paid yet. The ID field is mostly useful for locating
the specific database record if a duplicate invoice number exists.


\subsubsection{AP Aging Report}

This report can tell you how many invoices are past due and by how
much.

A summary report just shows vendors while a detail report shows individual
invoices.


\subsubsection{Tax Paid and Non-taxable Report}

These reports have known issues.  It is better to use the GL reports and filter 
accordingly.

\subsection{Vendor Reporting}


\subsubsection{Vendor Search}

The Vendor Search screen can be used to locate vendors or AP transactions
associated with those vendors.

The basic types of reports are:

\begin{description}
\item [{All}] Lists all vendors 
\item [{Active}] Lists those vendors currently active 
\item [{Inactive}] Lists those vendors who are currently inactive. time
frame. 
\item [{Orphaned}] Lists those vendors who do not have transactions associated
with them. These vendors can be deleted. 
\end{description}
One can include purchase orders, Requests for Quotations, AP invoices,
and AP transactions on this report as well if they occur between the
from and to dates.


\subsubsection{Vendor History}

This report can be used to obtain information about the past goods
and services ordered or received from vendors. One can find quantities,
partnumber, and sell prices on this report. This facility can be used
to search RFQ's, Purchase Orders, and AP Invoices.


\section{AR}


\subsection{Customers}

Customers are entered in using the AR-\textgreater Customers-\textgreater
Add Customer menu.

The salesperson is autopopulated with the current user who is logged
in. Otherwise, it looks fairly similar to the Vendor input screen.
Customers, like vendors can be assigned languages, but it is more
important to do so because invoices will be printed and sent to them.

The credit limit field can be used to assign an amount that one is
willing to do for a customer on credit.


\subsubsection{Customer Price Matrix}

The price list button can be used to enter specific discounts to the
customer, and groups of customers can be assigned a pricegroup for
the purpose of offering specific discounts on specific parts to the
customer. Such discounts can be temporary or permanent.


\subsection{AR Transactions}

AR Transactions are where one can add moneys owed the business by
customers. One can associate these transactions with income accounts,
and add payments if the item is paid when the invoice is issued.

The PO number field is used to track the PO that the customer sent.
This makes it easier to find items when a customer is asking for clarification
on a bill, for example.


\subsection{AR Invoices}

AR Invoices are designed to provide for the delivery of goods and
services to customers. One would normally issue these invoices at
the time when the everything has been done that is necessary to get
paid by the customer.

As with AP invoices, one can search for matches to partial part numbers
and descriptions, and enter initial payments at this screen.


\subsection{Cash Receipt}

The Cash-\textgreater Receipt screen allows you to accept prepayments
from customers or pay single or multiple invoices after they have
been posted. One can print a receipt, however the current templates
seem to be based on check printing templates and so are unsuitable
for this purpose. This presents a great opportunity for improvement.


\subsubsection{Cash Receipts for multiple customers}

The cash-\textgreater receipts screen allows you to accept payments
on all open customer invoices of all customers at once. One could
print (directly to a printer only) all receipts to be sent out if
this was desired.


\subsection{AR Transaction Reporting}

The AR Outstanding report is almost identical to the AP Outstanding
report and is not covered in any detail in this document.


\subsubsection{AR Transactions Report}

This is almost identical to the AP Transactions Report.

If a customer's PO has been associated with this transaction, one
can search under this field as well.


\subsubsection{AR Aging Report}

This report is almost identical to the AP Aging report, with the exception
that one can print up statements for customer accounts that are overdue.
One more application is to calculate interest based on balance owed
so that these can be entered as AR transactions associated with the
customer.


\subsection{Customer Reporting}

These reports are almost identical to the AP Vendor reports and are
not discussed in these notes.


\section{Projects}


\subsection{Project Basics}

A project is a logical collection of AR and AP transactions, orders,
and the like that allow one to better manage specific service or product
offerings. Ledger-SMB does not offer comprehensive project management
capabilities, and projects are only used here as they relate to accounting.

One can also add translated descriptions to the project names as well.


\subsection{Timecards}

Timecards allow one to track time entered on specific services. These
can then be used to generate invoices for the time entered.

The non-chargeable is the number of hours that are not billed on the
invoice.

One can then generate invoices based on this information.

The project field is not optional.


\subsection{Projects and Invoices}

One can select the project id for line items of both AR and AP invoices.
These will then be tracked against the project itself.


\subsection{Reporting}


\subsubsection{Timecard Reporting}

The Timecard Report allows one to search for timecards associated
with one or more projects. One can then use the total time in issuing
invoices (this is not automated yet).


\subsubsection{Project Transaction Reporting}

The Standard or GIFI options can be used to create different reports
(for example, for Canadian Tax reporting purposes).

This report brings up a summary that looks sort of like a chart of
accounts. Of one clicks on the account numbers, one can see the transactions
associated with the project.


\subsubsection{List of Projects}

This provides a simple way of searching for projects to edit or modify.


\subsection{Possibilities for Using Projects}

\begin{itemize}
\item One can use them similar to departments for tracking work done for
a variety of customers. 
\item One can use them for customer-specific projects, such as this training. 
\end{itemize}

\section{Quotations and Order Management}

This unit will introduce the business processes that Ledger-SMB allows.
These processes are designed to allow various types of businesses
to manage their orders and allow for rudimentary customer relationship
management processes to be built around this software. In this section,
we will introduce the work flow options that many businesses may use
in their day-to-day use of the software.


\subsection{Sales Orders}

Sales orders represent orders from customers that have not been delivered
or shipped yet. These orders can be for work in the future, for
back ordered products, or work in progress. A sales order can be generated
form an AR invoice or from a quotation automatically.


\subsection{Quotations}

Quotations are offers made to a customer but to which the customer
has not committed to the work. Quotations can be created from Sales
orders or AR Invoice automatically.


\subsection{Shipping}

The Shipping module (Shipping-\textgreater Shipping) allows one to
ship portions or entireties of existing sales orders, printing pick
lists and packing slips.

One can then generate invoices for those parts that were shipped.

In general, one will be more likely to use these features if they
have multiple warehouses that they ship from. More likely most customers
will just generate invoices from orders.


\subsection{AR Work Flow}


\subsubsection{Service Example}

A customer contacts your firm and asks for a quote on some services.
Your company would create a quotation for the job and email it to
the customer or print it and mail it. Once the customer agrees to
pay, one creates a sales order from the quotation.

When the work is completed, the sales order is converted into a sales 
invoice and this is presented to the customer as a bill.

Note that in some cases, this procedure may be shortened. If the customer
places an order without asking for a quotation and is offered a verbal
quote, then one might merely prepare the sales order.

%
\begin{figure}[hbtp]
 


\caption{Simple AR Service Invoice Workflow Example}

\setlength{\unitlength}{3947sp}%
%
\begingroup\makeatletter\ifx\SetFigFont\undefined%
\gdef\SetFigFont#1#2#3#4#5{%
  \reset@font\fontsize{#1}{#2pt}%
  \fontfamily{#3}\fontseries{#4}\fontshape{#5}%
  \selectfont}%
\fi\endgroup%
\begin{picture}(2674,4374)(5939,-4273)
\thinlines
{\color[rgb]{0,0,0}\put(7276,-736){\vector( 0,-1){1050}}
}%
{\color[rgb]{0,0,0}\put(7276,-2536){\vector( 0,-1){825}}
}%
{\color[rgb]{0,0,0}\put(5951,-736){\framebox(2650,825){}}
}%
{\color[rgb]{0,0,0}\put(5951,-2536){\framebox(2650,750){}}
}%
{\color[rgb]{0,0,0}\put(5951,-4261){\framebox(2650,900){}}
}%
\put(7201,-2161){\makebox(0,0)[b]{\smash{\SetFigFont{12}{14.4}{\rmdefault}{\mddefault}{\updefault}{\color[rgb]{0,0,0}Sales Order}%
}}}
\put(7201,-3811){\makebox(0,0)[b]{\smash{\SetFigFont{12}{14.4}{\rmdefault}{\mddefault}{\updefault}{\color[rgb]{0,0,0}AR Invoice}%
}}}
\put(7276,-361){\makebox(0,0)[b]{\smash{\SetFigFont{12}{14.4}{\rmdefault}{\mddefault}{\updefault}{\color[rgb]{0,0,0}Quotation}%
}}}
\end{picture}
 
\end{figure}



\subsubsection{Single Warehouse Example}

A customer contacts your firm and asks for a quotation for shipping
a part. You would create the quotation and when you get confirmation,
convert it to an order. Once the parts are in place you could go to
shipping and ship the part.

The billing department can then generate the invoice from the sales
order based on what merchandise has been shipped and mail it to the
customer.

Note that this requires that you have the part in your inventory.

%
\begin{figure}[hbtp]
 


\caption{AR Workflow with Shipping}

\setlength{\unitlength}{3947sp}%
%
\begingroup\makeatletter\ifx\SetFigFont\undefined%
\gdef\SetFigFont#1#2#3#4#5{%
  \reset@font\fontsize{#1}{#2pt}%
  \fontfamily{#3}\fontseries{#4}\fontshape{#5}%
  \selectfont}%
\fi\endgroup%
\begin{picture}(2424,6024)(4789,-6073)
\thinlines
{\color[rgb]{0,0,0}\put(4801,-886){\framebox(2400,825){}}
}%
{\color[rgb]{0,0,0}\put(4801,-2536){\framebox(2400,825){}}
}%
{\color[rgb]{0,0,0}\put(4801,-4261){\framebox(2400,825){}}
}%
{\color[rgb]{0,0,0}\put(4801,-6061){\framebox(2400,900){}}
}%
{\color[rgb]{0,0,0}\put(6001,-886){\vector( 0,-1){825}}
}%
{\color[rgb]{0,0,0}\put(6001,-2536){\vector( 0,-1){900}}
}%
{\color[rgb]{0,0,0}\put(6001,-4261){\vector( 0,-1){900}}
}%
\put(6001,-511){\makebox(0,0)[b]{\smash{\SetFigFont{12}{14.4}{\rmdefault}{\mddefault}{\updefault}{\color[rgb]{0,0,0}Quotation}%
}}}
\put(6001,-2161){\makebox(0,0)[b]{\smash{\SetFigFont{12}{14.4}{\rmdefault}{\mddefault}{\updefault}{\color[rgb]{0,0,0}Sales Order}%
}}}
\put(6001,-3886){\makebox(0,0)[b]{\smash{\SetFigFont{12}{14.4}{\rmdefault}{\mddefault}{\updefault}{\color[rgb]{0,0,0}Shipping}%
}}}
\put(6001,-5611){\makebox(0,0)[b]{\smash{\SetFigFont{12}{14.4}{\rmdefault}{\mddefault}{\updefault}{\color[rgb]{0,0,0}AR Invoice}%
}}}
\end{picture}
 
\end{figure}



\subsubsection{Multiple Warehouse Example}

A customer contacts your firm and asks for a quotation for a number
of different parts. You would create a quotation and when you get
confirmation, convert it to a sales order. When you go to ship the item,
you would select the warehouse in the drop-down menu, and select the
parts to ship. One would repeat with other warehouses until the entire
order is shipped.

Then the billing department would go to the sales order and generate
the invoice. It would then be mailed to the customer.

%
\begin{figure}[hbtp]
 


\caption{Complex AR Workflow with Shipping}

\setlength{\unitlength}{3947sp}%
%
\begingroup\makeatletter\ifx\SetFigFont\undefined%
\gdef\SetFigFont#1#2#3#4#5{%
  \reset@font\fontsize{#1}{#2pt}%
  \fontfamily{#3}\fontseries{#4}\fontshape{#5}%
  \selectfont}%
\fi\endgroup%
\begin{picture}(8874,6099)(1639,-6148)
\thinlines
{\color[rgb]{0,0,0}\put(4801,-886){\framebox(2400,825){}}
}%
{\color[rgb]{0,0,0}\put(4801,-2536){\framebox(2400,825){}}
}%
{\color[rgb]{0,0,0}\put(4801,-4261){\framebox(2400,825){}}
}%
{\color[rgb]{0,0,0}\put(4801,-6061){\framebox(2400,900){}}
}%
{\color[rgb]{0,0,0}\put(6001,-886){\vector( 0,-1){825}}
}%
{\color[rgb]{0,0,0}\put(6001,-2536){\vector( 0,-1){900}}
}%
{\color[rgb]{0,0,0}\put(6001,-4261){\vector( 0,-1){900}}
}%
{\color[rgb]{0,0,0}\put(8101,-4261){\framebox(2000,825){}}
}%
{\color[rgb]{0,0,0}\put(8101,-6136){\framebox(2000,900){}}
}%
{\color[rgb]{0,0,0}\put(1651,-4261){\framebox(2400,825){}}
}%
{\color[rgb]{0,0,0}\put(5251,-2536){\line( 0,-1){375}}
\put(5251,-2911){\line(-1, 0){2250}}
\put(3001,-2911){\vector( 0,-1){525}}
}%
{\color[rgb]{0,0,0}\put(6676,-2536){\line( 0,-1){375}}
\put(6676,-2911){\line( 1, 0){2550}}
\put(9226,-2911){\vector( 0,-1){525}}
}%
{\color[rgb]{0,0,0}\put(9226,-4261){\vector( 0,-1){975}}
}%
{\color[rgb]{0,0,0}\put(3001,-4261){\line( 0,-1){1275}}
\put(3001,-5536){\vector( 1, 0){1800}}
}%
\put(6001,-511){\makebox(0,0)[b]{\smash{\SetFigFont{12}{14.4}{\rmdefault}{\mddefault}{\updefault}{\color[rgb]{0,0,0}Quotation}%
}}}
\put(6001,-2161){\makebox(0,0)[b]{\smash{\SetFigFont{12}{14.4}{\rmdefault}{\mddefault}{\updefault}{\color[rgb]{0,0,0}Sales Order}%
}}}
\put(6001,-3886){\makebox(0,0)[b]{\smash{\SetFigFont{12}{14.4}{\rmdefault}{\mddefault}{\updefault}{\color[rgb]{0,0,0}Shipping}%
}}}
\put(6001,-5611){\makebox(0,0)[b]{\smash{\SetFigFont{12}{14.4}{\rmdefault}{\mddefault}{\updefault}{\color[rgb]{0,0,0}AR Invoice}%
}}}
\put(9301,-3886){\makebox(0,0)[b]{\smash{\SetFigFont{12}{14.4}{\rmdefault}{\mddefault}{\updefault}{\color[rgb]{0,0,0}Shipping}%
}}}
\put(9301,-5761){\makebox(0,0)[b]{\smash{\SetFigFont{12}{14.4}{\rmdefault}{\mddefault}{\updefault}{\color[rgb]{0,0,0}AR Invoice}%
}}}
\put(2776,-3886){\makebox(0,0)[b]{\smash{\SetFigFont{12}{14.4}{\rmdefault}{\mddefault}{\updefault}{\color[rgb]{0,0,0}Shipping}%
}}}
\end{picture}
 
\end{figure}



\subsection{Requests for Quotation (RFQ)}

A request for quotation would be a formal document one might submit
to a vendor to ask for a quote on a product or service they might
offer. These can be generated from Purchase Orders or AP Invoices.


\subsection{Purchase Orders}

A purchase order is a confirmation that is issued to the vendor to
order the product or service. Many businesses will require a purchase
order with certain terms in order to begin work on a product. These
can be generated from RFQ's or AP Invoices.


\subsection{Receiving}

The Shipping-\textgreater Receiving screen allows you to track the
parts received from an existing purchase order. Like shipping, it
does not post an invoice but tracks the received parts in the order.


\subsection{AP Work Flow}


\subsubsection{Bookkeeper entering the received items, order completed in full}

Your company inquires about the price of a given good or service from
another firm. You submit an RFQ to the vendor, and finding that the
price is reasonable, you convert it to an order, adjust the price
to what they have quoted, and save it. When the goods are delivered
you convert the order into an AP invoice and post it.

%
\begin{figure}[hbtp]
 


\caption{Simple AP Workflow}

\setlength{\unitlength}{3947sp}%
%
\begingroup\makeatletter\ifx\SetFigFont\undefined%
\gdef\SetFigFont#1#2#3#4#5{%
  \reset@font\fontsize{#1}{#2pt}%
  \fontfamily{#3}\fontseries{#4}\fontshape{#5}%
  \selectfont}%
\fi\endgroup%
\begin{picture}(2674,4374)(5939,-4273)
\thinlines
{\color[rgb]{0,0,0}\put(7276,-736){\vector( 0,-1){1050}}
}%
{\color[rgb]{0,0,0}\put(7276,-2536){\vector( 0,-1){825}}
}%
{\color[rgb]{0,0,0}\put(5951,-736){\framebox(2650,825){}}
}%
{\color[rgb]{0,0,0}\put(5951,-2536){\framebox(2650,750){}}
}%
{\color[rgb]{0,0,0}\put(5951,-4261){\framebox(2650,900){}}
}%
\put(7201,-2161){\makebox(0,0)[b]{\smash{\SetFigFont{12}{14.4}{\rmdefault}{\mddefault}{\updefault}{\color[rgb]{0,0,0}Purchase Order}%
}}}
\put(7201,-3811){\makebox(0,0)[b]{\smash{\SetFigFont{12}{14.4}{\rmdefault}{\mddefault}{\updefault}{\color[rgb]{0,0,0}AP Invoice}%
}}}
\put(7276,-361){\makebox(0,0)[b]{\smash{\SetFigFont{12}{14.4}{\rmdefault}{\mddefault}{\updefault}{\color[rgb]{0,0,0}RFQ}%
}}}
\end{picture}
 
\end{figure}



\subsubsection{Bookkeeper entering received items, order completed in part}

Your company inquires about the price of a given good or service from
another firm, You submit an RFQ to the vendor, and finding that the
price is acceptable, you convert it into an order, adjusting the price
to what they have quoted, and save it. When some of the goods are
received, you open up the purchase order, enter the number of parts
received, convert that order into an invoice, and post it. Repeat
until all parts are received.

%
\begin{figure}[hbtp]
 


\caption{AP Workflow with Receiving}

\setlength{\unitlength}{3947sp}%
%
\begingroup\makeatletter\ifx\SetFigFont\undefined%
\gdef\SetFigFont#1#2#3#4#5{%
  \reset@font\fontsize{#1}{#2pt}%
  \fontfamily{#3}\fontseries{#4}\fontshape{#5}%
  \selectfont}%
\fi\endgroup%
\begin{picture}(2424,6024)(4789,-6073)
\thinlines
{\color[rgb]{0,0,0}\put(4801,-886){\framebox(2400,825){}}
}%
{\color[rgb]{0,0,0}\put(4801,-2536){\framebox(2400,825){}}
}%
{\color[rgb]{0,0,0}\put(4801,-4261){\framebox(2400,825){}}
}%
{\color[rgb]{0,0,0}\put(4801,-6061){\framebox(2400,900){}}
}%
{\color[rgb]{0,0,0}\put(6001,-886){\vector( 0,-1){825}}
}%
{\color[rgb]{0,0,0}\put(6001,-2536){\vector( 0,-1){900}}
}%
{\color[rgb]{0,0,0}\put(6001,-4261){\vector( 0,-1){900}}
}%
\put(6001,-2161){\makebox(0,0)[b]{\smash{\SetFigFont{12}{14.4}{\rmdefault}{\mddefault}{\updefault}{\color[rgb]{0,0,0}Purchase Order}%
}}}
\put(6001,-511){\makebox(0,0)[b]{\smash{\SetFigFont{12}{14.4}{\rmdefault}{\mddefault}{\updefault}{\color[rgb]{0,0,0}RFQ}%
}}}
\put(6001,-3886){\makebox(0,0)[b]{\smash{\SetFigFont{12}{14.4}{\rmdefault}{\mddefault}{\updefault}{\color[rgb]{0,0,0}Receiving}%
}}}
\put(6001,-5611){\makebox(0,0)[b]{\smash{\SetFigFont{12}{14.4}{\rmdefault}{\mddefault}{\updefault}{\color[rgb]{0,0,0}AP Invoice}%
}}}
\end{picture}
 
\end{figure}



\subsubsection{Receiving staff entering items}

Your company inquires about the price of a given good or service from
another firm, You submit an RFQ to the vendor, and finding that the
price is acceptable, you convert it into an order, adjusting the price
to what they have quoted, and save it. When some or all of the goods
are received, the receiving staff goes to Shipping-Receiving, locates
the purchase order, and fills in the number of items received.

The bookkeeper can then determine when all items have been received
and post the invoice at that time.

%
\begin{figure}[hbtp]
 


\caption{Complex AP Workflow}

\setlength{\unitlength}{3947sp}%
%
\begingroup\makeatletter\ifx\SetFigFont\undefined%
\gdef\SetFigFont#1#2#3#4#5{%
  \reset@font\fontsize{#1}{#2pt}%
  \fontfamily{#3}\fontseries{#4}\fontshape{#5}%
  \selectfont}%
\fi\endgroup%
\begin{picture}(8874,6099)(1639,-6148)
\thinlines
{\color[rgb]{0,0,0}\put(4801,-886){\framebox(2400,825){}}
}%
{\color[rgb]{0,0,0}\put(4801,-2536){\framebox(2400,825){}}
}%
{\color[rgb]{0,0,0}\put(4801,-4261){\framebox(2400,825){}}
}%
{\color[rgb]{0,0,0}\put(4801,-6061){\framebox(2400,900){}}
}%
{\color[rgb]{0,0,0}\put(6001,-886){\vector( 0,-1){825}}
}%
{\color[rgb]{0,0,0}\put(6001,-2536){\vector( 0,-1){900}}
}%
{\color[rgb]{0,0,0}\put(6001,-4261){\vector( 0,-1){900}}
}%
{\color[rgb]{0,0,0}\put(8101,-4261){\framebox(2000,825){}}
}%
{\color[rgb]{0,0,0}\put(8101,-6136){\framebox(2000,900){}}
}%
{\color[rgb]{0,0,0}\put(1651,-4261){\framebox(2400,825){}}
}%
{\color[rgb]{0,0,0}\put(5251,-2536){\line( 0,-1){375}}
\put(5251,-2911){\line(-1, 0){2250}}
\put(3001,-2911){\vector( 0,-1){525}}
}%
{\color[rgb]{0,0,0}\put(6676,-2536){\line( 0,-1){375}}
\put(6676,-2911){\line( 1, 0){2550}}
\put(9226,-2911){\vector( 0,-1){525}}
}%
{\color[rgb]{0,0,0}\put(9226,-4261){\vector( 0,-1){975}}
}%
{\color[rgb]{0,0,0}\put(3001,-4261){\line( 0,-1){1275}}
\put(3001,-5536){\vector( 1, 0){1800}}
}%
\put(6001,-2161){\makebox(0,0)[b]{\smash{\SetFigFont{12}{14.4}{\rmdefault}{\mddefault}{\updefault}{\color[rgb]{0,0,0}Purchase Order}%
}}}
\put(6001,-511){\makebox(0,0)[b]{\smash{\SetFigFont{12}{14.4}{\rmdefault}{\mddefault}{\updefault}{\color[rgb]{0,0,0}RFQ}%
}}}
\put(2776,-3886){\makebox(0,0)[b]{\smash{\SetFigFont{12}{14.4}{\rmdefault}{\mddefault}{\updefault}{\color[rgb]{0,0,0}Receiving}%
}}}
\put(6001,-3886){\makebox(0,0)[b]{\smash{\SetFigFont{12}{14.4}{\rmdefault}{\mddefault}{\updefault}{\color[rgb]{0,0,0}Receiving}%
}}}
\put(9301,-3886){\makebox(0,0)[b]{\smash{\SetFigFont{12}{14.4}{\rmdefault}{\mddefault}{\updefault}{\color[rgb]{0,0,0}Receiving}%
}}}
\put(6001,-5611){\makebox(0,0)[b]{\smash{\SetFigFont{12}{14.4}{\rmdefault}{\mddefault}{\updefault}{\color[rgb]{0,0,0}AP Invoice}%
}}}
\put(9301,-5761){\makebox(0,0)[b]{\smash{\SetFigFont{12}{14.4}{\rmdefault}{\mddefault}{\updefault}{\color[rgb]{0,0,0}AP Invoice}%
}}}
\end{picture}
 
\end{figure}



\subsection{Generation and Consolidation}


\subsubsection{Generation}

The Generation screen allows you to generate Purchase Orders based
on sales orders. One selects the sales orders one wants to use, and
clicks \char`\"{}Generate Purchase Orders.\char`\"{} Then one selects
clicks on the parts to order, adjusts the quantity if necessary, and
clicks \char`\"{}Select Vendor.\char`\"{} This process is repeated
for every vendor required. Then the Generate Orders button is clicked.


\subsubsection{Consolidation}

One can consolidate sales and/or purchase orders using this screen.
For the consolidation to work you must have more than one order associated
with the relevant customer or vendor.


\subsection{Reporting}

The reporting functionality in the order management is largely limited
to the ability to locate purchase orders, sales orders, RFQ's, and
quotations.


\subsection{Shipping Module: Transferring Inventory between Warehouses}

One can transfer inventory between warehouses if necessary by using
the Shipping-\textgreater Transfer Inventory screen.


\section{HR}

The HR module is currently limited to tracking employees for and their
start and end dates. It has very little other functionality. One could
build payroll systems that could integrate with it however.


\section{POS}

LedgerSMB 1.2 includes a number of components merged from Metatron Technology 
Consulting's SL-POS.  Although it is still not a perfect solution, it is greatly improved in both workflow and hardware support.  It is suitable for retail 
establishments at the moment.

\subsection{Sales Screen}

The sales screen looks very much like a normal invoice entry screen
with a few differences.

\begin{itemize}
\item The discount text field is not available, nor is the unit field.. 
\item The next part number is automatically focused when the data loads
for rapid data entry. 
\item Hot keys for the buttons are Alt-U for update, Alt-P for print, Alt-O
for post, and Alt-R for print and post. 
\item Part Groups appear at the bottom of the screen. 
\item Alt-N moves the cursor to the next free payment line.
\end{itemize}

\subsection{Possibilities for Data Entry}

\begin{itemize}
\item Barcode scanners can be used to scan items in as they are being rung
in. 
\item One could use touch screens, though this would ideally require some
custom stylesheets to make it efficient. 
\end{itemize}

\subsection{Hardware Support}

As LedgerSMB is a web-based application, the web browser usually
does not allow the page to write to arbitrary files. Therefore hardware
support for pole displays, etc. is not readily possible from the application
itself. LedgerSMB gets around this limitation by using an additional set of 
network sockets from the server to the client to control its hardware.  This 
naturally requires that other software is also running on the client.

Notes for specific types of hardware are as follows:

\begin{description}
\item [{Touch}] screens: The default stylesheet is not really usable from
a touchscreen as the items are often too small. One would need to
modify the stylesheets to ensure that the relevant items would be
reasonable. Setting down the resolution would also help. 
\item [{Receipt}] Printers: ESC/POS printers generally work in text mode.
Control sequences can be embedded in the template as necessary. 
\item [{Pole}] Displays: Generally supported.  Only the Logic Controls PD3000 is
supported out of the box, but making this work for other models ought to be
trivial.
\item [{Cash}] Drawers: These should be attached to the printer. The control
codes is then specified in the pos.conf.pl so that the command is sent to the
printer when the open till button is pushed.
\item [{Barcode}] Scanners: Most customers use decoded barcode scanners
through a keyboard wedge interface. This allows them to scan items
as if they were typing them on the keyboard. 
\end{description}

\subsection{Reports}


\subsubsection{Open Invoices}

The POS-\textgreater Open screen allows one to find any POS receipts
that are not entirely paid off.


\subsubsection{Receipts}

The POS-\textgreater Receipts screen allows one to bring up a basic
record of the POS terminals. It is not sufficient for closing the
till, however, though it may help for reconciliation.

The till column is the last component or octet of the terminal's IP
address. Therefore it is a good idea to try to avoid having IP addresses
where the last octet is the same.

All entries are grouped by date and source in this report.


\section{General Ledger}


\subsection{GL Basics}

The General Ledger is the heart of Ledger-SMB. Indeed, Ledger-SMB
is designed to be as close as possible to a software equivalent of
a paper-based accounting program (but with no difference between the
General Ledger and General Journal).


\subsubsection{Paper-based accounting systems and the GL}

In order to understand the principle of the General Ledger, one must
have a basic understanding of the general process of bookkeeping using
double-entry paper-based accounting systems.

Normally when a transaction would be recorded, it would first be recorded
in the \char`\"{}General Journal\char`\"{} which would contain detailed
information about the transaction, notes, etc. Then the entries from
the General Journal would be transcribed to the General Ledger, where
one could keep closer tabs on what was going on in each account.

In the general journal, all transactions are listed chronologically
with whatever commentary is deemed necessary, while in the general
ledger each account has its own page and transactions are recorded
in a simple and terse manner. The General Journal is the first place
the transaction is recorded and the General Ledger is the last.

At the end of the accounting period, the GL transactions would be
summarized into a trial balance and this would be used for creating
financial statements and closing the books at the end of the year.


\subsubsection{Double Entry Examples on Paper}

Let us say that John starts his business with an initial investment
of \$10,000.

This is recorded in the General Journal as follows (in this example,
suppose it is page 1):

\begin{tabular}{|l|l|l|r|r|}
\hline 
Date &
Accounts and Explanation &
Ref &
DEBIT &
CREDIT \tabularnewline
\hline 
March 1 &
Checking Account &
1060 &
10000.00 &
\tabularnewline
&
John Doe Capital &
3011 &
&
10000.00\tabularnewline
&
John Doe began a business &
&
&
\tabularnewline
&
with an investment of &
&
&
\tabularnewline
&
\$10000 &
&
&
\tabularnewline
\hline
\end{tabular}\medskip{}


This would then be transcribed into two pages of the General Ledger.
The first page might be the Checking Account page:\medskip{}


\begin{tabular}{|l|l|l|r|l|l|l|r|}
\hline 
DATE &
EXPLANATION &
REF. &
DEBITS &
DATE &
EXPLANATION &
REF. &
CREDITS\tabularnewline
\hline 
March 1 &
&
J1 &
10000.00 &
&
&
&
\tabularnewline
\hline
\end{tabular}\medskip{}


On the John Doe Capital page, we would add a similar entry:\medskip{}


\begin{tabular}{|l|l|l|r|l|l|l|r|}
\hline 
DATE &
EXPLANATION &
REF. &
DEBITS &
DATE &
EXPLANATION &
REF. &
CREDITS\tabularnewline
\hline 
&
&
&
&
March 1 &
&
J1 &
10000.00\tabularnewline
\hline
\end{tabular}\medskip{}



\subsubsection{The GL in Ledger-SMB}

The paper-based accounting procedure works well when one is stuck
with paper recording requirements but it has one serious deficiency---
all of this transcribing creates an opportunity for errors.

Relational databases relieve the need for such transcription as it
is possible to store everything physically in a way similar to the
way a General Journal is used in the paper-based systems and then
present the same information in ways which are more closely related
to the General Ledger book.

This is the exact way that the General Ledger is used in Ledger-SMB.
The actual data is entered and stored as if it was a general journal,
and then the data can be presented in any number of different ways.

All modules of Ledger-SMB that involve COA accounts store their data
in the General Ledger (it is a little more complex than this but this
is very close to the actual mechanism).


\subsection{Cash Transfer}

The simplest form of GL entry in Ledger-SMB is the Cash-\textgreater
Transfer screen. This screen shows two transaction lines, and fields
for reference, department, description, and notes.

The field descriptions are as follows:

\begin{description}
\item [{Reference}] refers to the source document for the transfer. One
can use transfer sheets, bank receipt numbers, etc for this field. 
\item [{Description}] is optional but really should be filled in. It ought
to be a description of the transaction. 
\item [{Notes}] provide supplemental information for the transaction. 
\item [{FX}] indicates whether foreign exchange is a factor in this transaction. 
\item [{Debit}] indicates money going \textbf{into} the asset account. 
\item [{Credit}] indicates money coming \textbf{out} of the asset account. 
\item [{Source}] is the source document for that portion of the transaction. 
\item [{Memo}] lists additional information as necessary.
\item [{Project}] allows you to assign this line to a project. 
\end{description}
The credit and debit options seem to be the opposite of what one would
think of concerning one's bank account. The reason is that your bank
statement is done from the bank's point of view.  Your bank account balance
is an asset to you and therefor you show it as having a debit balance, but
to the bank it is money they owe you and so they show it as having a credit
balance.

Note that in this screen, when an item is updated, it will reduce
the number of lines to those already filled in plus an extra line
for the new line in the data entry.


\subsection{GL Transactions}

The GL Transaction screen (General Ledger-\textgreater Add Transaction)
is identical to the Cash Transfer screen with the exception that it
starts with nine lines instead of two. Otherwise, they are identical.

Again, one must be careful with debits and credits. Often it is easy
to get confused. It is generally worth while to go back to the principle
that one tracks them with regard to their impact on the equity accounts.
So expenses are credits because they debit the equity accounts, and
income is a debit because it credits the retained earning equity account.


\subsection{Payroll as a GL transaction}

Currently payroll must be done as a GL transaction. The attempts to
create a payroll system that would ship with LSMB have largely stalled.

Most customers running their businesses will have an idea of how to
do this.

%
\begin{figure}[hbtp]
 


\caption{Payroll as a GL Transaction (Purely fictitious numbers)}

\begin{tabular}{|l|r|r|}
\hline 
Account &
Debit &
Credit \tabularnewline
5101 Wages and Salaries &
500 &
\tabularnewline
2032 Accrued Wages &
&
450 \tabularnewline
2033 Fed. Income Tax wthd &
&
30 \tabularnewline
2034 State Inc. Tax. wthd &
&
15 \tabularnewline
2035 Social Security wthd &
&
3 \tabularnewline
2036 Medicare wthd &
&
2 \tabularnewline
2032 Accrued Wages &
450 &
\tabularnewline
1060 Checking Acct &
&
450 \tabularnewline
\hline
\end{tabular}
\end{figure}



\subsection{Reconciliation}

To reconcile an account (say, when one would get a checking account
statement), one would go to cash/reconciliation, and check off the
items that have cleared. One can then attempt to determine where any
errors lie by comparing the total on the statement with the total
that LSMB generates.

This can be done for other accounts too, such as petty cash.%
\footnote{Petty cash denotes a drawer of cash that is used to pay small expenses.
When an expense is paid, it is recorded on a slip of paper that is
stored for reconciliation purposes.%
}


\subsection{Reports}

The most flexible report in Ledger-SMB is the GL report because it
has access to the entire set of financial transactions of a business.
Every invoice posted, payment made or received, etc. can be located
here.

The search criteria include:

\begin{description}
\item [{Reference}] is the invoice number, or other reference number associated
with the transaction. 
\item [{Source}] is the field related to the source document number in
a payment or other transaction.%
\footnote{Source documents are things like receipts, canceled checks, etc. that
can be used to verify the existence and nature of a transaction.%
} 
\item [{Memo}] relates to the memo field on a payment.
\item [{Department}] can be used to filter results by department. 
\item [{Account}] Type can be used to filter results by type of account
(Asset, Liability, etc.) 
\item [{Description}] can be used to filter by GL description or by
customer/vendor name. 
\end{description}
The actual format of the report looks more like what one would expect
in a paper accounting system's general journal than a general ledger
per se. A presentation of the data that is more like the paper general
ledger is found in the Chart of Accounts report.


\subsubsection{GL as access to almost everything else}

The GL reports can be used to do all manner of things. One can determine,
for example, which AP invoice or transaction was paid with a certain
check number or which invoice by a specific customer was paid by a specific
check number.


\section{Recurring Transactions}

Any transaction or invoice may be repeated a number of times in regular
intervals. To schedule any GL, AR, or AP transaction or invoice, click
the schedule button.

In general the reference number should be left blank as this will
force Ledger-SMB to create a new invoice or transaction number for
each iteration. The rest of the options are self-explanatory. Note
that a blank number if iterations will result in no recurrences of
the transaction.

To process the recurring transactions, click on the Recurring Transactions
option on the main menu select the ones you want to process and click
\char`\"{}Process Transactions.\char`\"{}


\section{Financial Statements and Reports}

Financial statements and reports are a very important part of any
accounting system. Accountants and business people rely on these reports
to determine the financial soundness of the business and its prospects
for the next accounting period.


\subsection{Cash v. Accrual Basis}

Financial statements, such as the Income Statement and Balance Sheet
can be prepared either on a cash or accrual basis. In cash-basis accounting,
the income is deemed earned when the customer pays it, and the expenses
are deemed incurred when the business pays them.

There are a number of problems with cash-basis accounting from a business
point of view. The most serious is that one can misrepresent the wellbeing
of a business by paying a large expense after a deadline. Thus cash-basis
accounting does not allow one to accurately pair the income with the
related expense as these are recorded at different times. If one cannot
accurately pair the income with the related expense, then financial
statements cannot be guaranteed to tell one much of anything about
the well-being of the business.

In accrual basis accounting, income is considered earned when the
invoice is posted, and expenses are considered incurred at the time
when the goods or services are delivered to the business. This way,
one can pair the income made from the sale of a product with the expense
incurred in bringing that product to sale. This pairing allows for
greater confidence in business reporting.


\subsection{Viewing the Chart of Accounts and Transactions}

The Reports--\textgreater Chart of Accounts will provide the chart
of accounts along with current totals in each account.

If you click on an account number, you will get a screen that allows
you to filter out transactions in that account by various criteria.
One can also include AR/AP, and Subtotal in the report.

The report format is similar to that of a paper-based general ledger.


\subsection{Trial Balance}


\subsubsection{The Paper-based function of a Trial Balance}

In paper-based accounting systems, the accountant at the end of the
year would total up the debits and credits in every account and transfer
them onto another sheet called the trial balance. The accountant would
check to determine that the total debits and credits were equal and
would then transfer this information onto the financial statements.
It was called a trial balance because it was the main step at which
the error-detection capabilities of double-entry accounting systems
were used.


\subsubsection{Running the Trial Balance Report}

This report is located under Reports --\textgreater Trial Balance.
One can filter out items by date, accounting period, or department.
One can run the report by accounts or using GIFI classifications to
group accounts together.

From this report, you can click on the account number and see all
transactions on the trial balance as well as whether or not they have
been reconciled.


\subsubsection{What if the Trial Balance doesn't Balance?}

If the trial balance does not balance, get technical support immediately.
This usually means that transactions were not entered properly. Some
may have been out of balance, or some may have gone into non-existent
accounts (believe it or not, Ledger-SMB does not check this latter
issue).


\subsubsection{Trial Balance as a Summary of Account Activity}

The trial balance offers a glance at the total activity in every account.
It can provide a useful look at financial activity at a glance for
the entire business.


\subsubsection{Trial Balance as a Budget Planning Tool}

By filtering out departments, one can determine what a department
earned and spent during a given financial interval. This can be used
in preparing budgets for the next accounting period.


\subsection{Income Statement}

The Income Statement is another tool that can be used to assist with
budgetary planning as well as provide information on the financial
health of a business.

The report is run from Reports--\textgreater Income Statement. The
report preparation screen shows the following fields:

\begin{description}
\item [{Department}] allows you to run reports for individual departments.
This is useful for budgetary purposes. 
\item [{Project}] allows you to run reports on individual projects. This
can show how profitable a given project was during a given time period. 
\item [{From}] and To allow you to select arbitrary from and to dates. 
\item [{Period}] allows you to specify a standard accounting period. 
\item [{Compare to}] fields allow you to run a second report for comparison
purposes for a separate range of dates or accounting period. 
\item [{Decimalplaces}] allows you to display numbers to a given precision. 
\item [{Method}] allows you to select between accrual and cash basis reports. 
\item [{Include}] in Report provides various options for reporting. 
\item [{Accounts}] allows you to run GIFI reports instead of the standard
ones. 
\end{description}
The report shows all income and expense accounts with activity during
the period when the report is run, the balances accrued during the
period, as well as the total income and expense at the bottom of each
section. The total expense is subtracted from the total income to
provide the net income during the period. If there is a loss, it appears
in parentheses.


\subsubsection{Uses of an Income Statement}

The income statement provides a basic snapshot of the overall ability
of the business to make money. It is one of the basic accounting statements
and is required, for example, on many SEC forms for publicly traded
firms.

Additionally, businessmen use the income statement to look at overall
trends in the ability of the business to make money. One can compare
a given month, quarter, or year with a year prior to look for trends
so that one can make adjustments in order to maximize profit.

Finally, these reports can be used to provide a look at each department's
performance and their ability to work within their budget. One can
compare a department or project's performance to a year prior and
look for patterns that can indicate problems or opportunities that
need to be addressed.


\subsection{Balance Sheet}

The balance sheet is the second major accounting statement supported
by Ledger-SMB. The balance sheet provides a snapshot of the current
financial health of the business by comparing assets, liabilities,
and equity.

In essence the balance sheet is a statement of the current state of
owner equity. Traditionally, it does not track changes in owner equity
in the same way the Statement of Owner Equity does.

The Balance Sheet report preparation screen is much simpler than the
Income Statement screen. Balance sheets don't apply to projects, but
they do apply to departments. Also, unlike an income statement, a
balance sheet is fixed for a specific date in time. Therefore one
does not need to select a period.

The fields in creating a balance sheet are:

\begin{description}
\item [{Department}] allows you to run separate balance sheets for each
department. 
\item [{As}] at specifies the date. If blank this will be the current date. 
\item [{Compare to}] specifies the date to compare the balance sheet to. 
\item [{Decimalplaces}] specifies the number of decimal places to use. 
\item [{Method}] selects between cash and accrual basis. 
\item [{Include}] in report allows you to select supplemental information
on the report. 
\item [{Accounts}] allows you to select between standard and GIFI reports. 
\end{description}
The balance sheet lists all asset, liability, and equity accounts
with a balance. Each category has a total listed, and the total of
the equity and liability accounts is also listed.

The total assets should be equal to the sum of the totals of the liability
and equity accounts.


\subsection{What if the Balance Sheet doesn't balance?}

Get technical support immediately, This may indicate that out of balance
transactions were entered or that transactions did not post properly.


\subsection{No Statement of Owner Equity?}

The Statement of Owner Equity is the one accounting statement that
Ledger-SMB does not support. However, it can be simulated by running
a balance sheet at the end of the time frame in question and comparing
it to the beginning. One can check this against an income statement
for the period in question to verify its accuracy. The statement of
owner equity is not as commonly used now as it once was.


\section{The Template System}

Ledger-SMB allows most documents to be generated according to a template
system. This allows financial statements, invoices, orders, and the
like to be customized to meet the needs of most businesses. Company
logos can be inserted, the format can be radically altered, one can
print letters to be included with checks to vendors instead of the
checks themselves, and the like. In the end, there is very little
that cannot be accomplished regarding modification of these documents
with the template system.

One can define different templates for different languages, so that
a customer in Spain gets a different invoice than a customer in Canada.


\subsection{Text Templates}

The only template that uses a text-only format is the POS receipt.
This example provides the simplest way to understand the template
system.

The first two lines are:\\
 \textless\%company align=center width=40\%\textgreater\\
 \textless\%address align=center width=40\%\textgreater\\


The first line tells Ledger-SMB to print the company name as passed
to it via a variable, centered, with a page width of 40 characters.
The second line does the same thing with the address.

These variables are usually passed to the invoice using form fields
(hidden or otherwise) in the submitting web page. The printing script,
however, can disable some of these fields or add others via database
lookups and the like.

In all types of templates, variable substitution occurs between \textless\%
and \%\textgreater. One can optionally specify an alignment or a
width but these are really only useful in text templates.


\subsection{HTML Templates}

The following templates exist in HTML format:

\begin{itemize}
\item Income Statement 
\item Balance Sheet 
\item Invoice (AR) 
\item AR Transaction 
\item AP Transaction 
\item Packing List 
\item Pick List 
\item Sales Order 
\item Work Order 
\item Purchase Order 
\item Bin List 
\item Statement 
\item Quotation 
\item RFQ 
\item Time Card 
\end{itemize}
These templates can be edited by an HTML editor. However, it is generally
recommended that one back up templates first. The reason is that some
HTML editors will fully re-parse the HTML and save it back without
what they see as invalid tags. Most editors, however, will save the
variable substitution tags because similar tags are also used by Microsoft's
active server pages.

Finally, some editors are known to mangle formatting, so many problems
can be avoided by ensuring that one has a backup of the templates,
especially if they have already been customized.


\subsection{\LaTeX{}\ Templates}

The following templates, by default, are available in \LaTeX{}\ :

\begin{itemize}
\item Invoice 
\item AR Transaction 
\item AP Transaction 
\item Packing List 
\item Pick List 
\item Sales Order 
\item Work Order 
\item Purchase Order 
\item Bin List 
\item Statement 
\item Check 
\item Receipt 
\item Quotation 
\item RFQ 
\item Time Card 
\end{itemize}
\LaTeX{}\ templates allow one to generate PDF and postscript documents
and print directly to a postscript-enabled printer or print software
(like CUPS).

\LaTeX{}\ templates can be edited using a standard text editor (like
vim or emacs), or using a synchronous \LaTeX{}\ implementation such
as \LyX{}.


\subsubsection{What is \LaTeX{}\ ?}

\LaTeX{}\ (pronounced LAY-tech) is an extension on the \TeX{}\ typesetting
system. It largely consists of a set of macros that allow one to focus
on the structure of the document while letting the \TeX{}\ engine
do the heavy lifting in terms of determining the optimal formatting
for the page. \LaTeX{}\ is used in a large number of academic journals
(including those of the American Mathematics Association). It is available
at \url{http://www.tug.org} and is included in most Linux distributions.

Like HTML, \LaTeX{}\ uses plain text documents to store the formatting
information and then when the document is rendered, attempts to fit
it onto a page. \LaTeX{}\
supports the concept of stylesheets, allowing one to separate content
from format, and this feature is used in many higher-end applications,
like journal publication.

Unlike HTML, \LaTeX{}\ is a complete though simple programming language
that allows one to redefine internals of the system for formatting
purposes.

This document is written in \LaTeX{}.


\subsubsection{Using \LyX{} to Edit \LaTeX{}\ Templates}

\LyX{} is a synchronous \LaTeX{}\ editor that runs on Windows, UNIX/Linux,
and Mac OS X. It requires an installed \LaTeX{}-2e implementation
and can be obtained at \url{http://www.lyx.org}. Like the most common
\LaTeX{}\ implementations, it is open source and is included with most 
Linux distributions.

\subsection{Customizing Logos}

\LaTeX{}\ requires different formats of logos depending on whether
the document is going to be generated as a PDF or as postscript. Postscript
requires an embedded postscript graphic, while PDF requires any type
of graphic other than embedded postscript. Usually one uses a PNG's
for PDF's, though GIF's could be used as well. The logo for a \LaTeX{}\ document
resides in the users directory.

HTML documents can have logos in many different formats. PNG's are
generally preferred for printing reasons. The image can be stored
anywhere and merely referenced in the HTML.

Note: Always test the an invoice with images to ensure that
the rest of the page format is not thrown off by it.


\subsection{How are They Stored in the Filesystem?}

The template directory (\char`\"{}templates\char`\"{} in the root
Ledger-SMB install directory) contains all the root templates used
by Ledger-SMB. These follow a naming convention of COAType-templatename.ext
where COAType is the type of dataset that was created when the user
was created, templatename is the name of the template, and ext is
either txt, html, or tex (for text, html, and \LaTeX{}\ respectively).

Inside this directory are one or more subdirectories where the relevant
templates have been copied as default language templates for the user.
Many users can use the same user directory (which bears the name of
the Ledger-SMB username). Within this directory are more subdirectories
for translated templates, one for each language created.


\subsection{Upgrade Issues}

When Ledger-SMB is upgraded, the templates are not replaced. This
is designed to prevent the upgrade script from overwriting changes
made during the course of customizing the templates.

Occasionally, however, the data model changes in a way which can cause
the templates to stop printing certain information. When information
that was showing up before an upgrade stops showing up, one can either
upgrade the templates by copying the source template over the existing
one, or one can edit the template to make the change.

\section{An Introduction to the CLI}

\subsection{Conventions}

The command-line API will be referred to as the API.

\subsection{Preliminaries}

Logging into Ledger-SMB (1.2+) updates a row in the users\_conf table
containing your account configuration defaults and other information. 
The implication for API users of LSMB is that you must login as part 
of running API scripts. For security purposes, it is recommended that
scripts prompt for a password rather than storing it.

All scripts included in the documentation can also be found in the doc/samples
directory.

Consider a simple example:

 cd /usr/local/ledger-smb
./ct.pl "login=name\&path=bin\&password=xxxxx\&action=search\&db=customer"

The cd command moves your terminal session's current working directory into
the main Ledger-SMB directory. Then the Ledger-SMB perl script ct.pl is called 
with one long line as an argument. The argument is really several variable=value pairs
separated by ampersands (\&). The value for the login variable is the username
that Ledger-SMB is to use, and the value for the password variable is the plaintext password.

To build our examples we will use a username of "clarkkent" who has a password
of "lOis,lAn3".

 cd /usr/local/ledger-smb
./ct.pl "login=clarkkent\&path=bin\&password=lOis,lAn3\&action=search\&db=customer"

If we execute these commands we will get the html for the search form for 
the customer database. This result isn't useful in itself, but it shows we 
are on the right track.


\subsection{First Script: lsmb01-cli-example.sh}

With a working example, we can start to build reproducible routines that we can grow
to do some useful work.

This is a bash script which:

 1. sets NOW to the current working directory
 2. prompts for and reads your Ledger-SMB login
 3. prompts for and reads (non-echoing) your Ledger-SMB password
 4. changes directory to /usr/local/ledger-smb
 5. constructs login and logout commands and a transaction command
 6. logins into ledger-smb (in a real program, output would be checked for
    success or failure)
 7. executes the transaction
 8. logs out of ledger-smb (although this is not necessary)
 9. returns to the original working directory
10. exits

Running lsmb01-cli-example.sh produces:

\$ lsmb01-cli-example.sh 

Ledger-SMB login: clarkkent

Ledger-SMB password: 

\begin{verbatim}
<body>

<form method=post action=ct.pl>

<input type=hidden name=db value=customer>

<table width=100%>
  <tr>
    <th class=listtop>Search</th>
.
.
.
\end{verbatim}

A script like this would work well for simple batch transactions, but
bash is not a very friendly language for application programming.


A nicer solution would be to use a language such as perl to drive the
command line API.

\subsubsection{Script 1 (Bash)}
\begin{verbatim}
#!/bin/bash
#######################################################################
#
# lsmb01-cli-example.sh
# Copyright (C) 2006. Louis B. Moore
#
# $Id: $
#
# This program is free software; you can redistribute it and/or modify
# it under the terms of the GNU General Public License as published by
# the Free Software Foundation; either version 2 of the License, or
# (at your option) any later version.
#
# This program is distributed in the hope that it will be useful,
# but WITHOUT ANY WARRANTY; without even the implied warranty of
# MERCHANTABILITY or FITNESS FOR A PARTICULAR PURPOSE.  See the
# GNU General Public License for more details.
# You should have received a copy of the GNU General Public License
# along with this program; if not, write to the Free Software
# Foundation, Inc., 675 Mass Ave, Cambridge, MA 02139, USA.
#
#######################################################################

NOW=`pwd`

echo -n "Ledger-SMB login: "
read LSLOGIN
echo

echo -n "Ledger-SMB password: "
stty -echo
read LSPWD
stty echo
echo

ARG="login=${LSLOGIN}&password=${LSPWD}&path=bin&action=search&db=customer"

LGIN="login=${LSLOGIN}&password=${LSPWD}&path=bin&action=login"
LGOT="login=${LSLOGIN}&password=${LSPWD}&path=bin&action=logout"

cd /usr/local/ledger-smb

./login.pl $LGIN 2>&1  > /dev/null
./ct.pl $ARG
./login.pl $LGOT 2>&1  > /dev/null

cd $NOW

exit 0
\end{verbatim}


\subsection{Second Script: lsmb02-cli-example.pl}

Our second script is written in perl and logs you in but it still uses the API 
in its simplest form, that is, it builds commands and then executes them. This 
type of script can be used for more complex solutions than the simple bash script 
above, though it is still fairly limited. If your needs require, rather than have 
the script build and then execute the commands it could be written to generate a 
shell script which is executed by hand. 

This script begins by prompting for your Ledger-SMB login and password. Using
the supplied values a login command is constructed and passed into the runLScmd
subroutine. runLScmd changes directory to /usr/local/ledger-smb/ for the length 
of the subroutine. It formats the command and executes it and returns both the 
output and error information to the caller in a scalar.

The script checks to see if there was an error in the login, exiting if there was.

Next, the script reads some records which are stored in the program following the
\_\_END\_\_ token. It takes each record in turn, formats it then feeds each transaction
through runLScmd and looks at the results for a string that signals success.

Once all the transactions are processed, runLScmd is called one last time to
logout and the script exits.

\subsubsection{Script 2 (Perl)}
\begin{verbatim}
#!/usr/bin/perl -w
#
#  File:         lsmb02-cli-example.pl
#  Environment:  Ledger-SMB 1.2.0+
#  Author:       Louis B. Moore
#
#  Copyright (C)   2006  Louis B. Moore
#
#  This program is free software; you can redistribute it and/or
#  modify it under the terms of the GNU General Public License
#  as published by the Free Software Foundation; either version 2
#  of the License, or (at your option) any later version.
#
#  This program is distributed in the hope that it will be useful,
#  but WITHOUT ANY WARRANTY; without even the implied warranty of
#  MERCHANTABILITY or FITNESS FOR A PARTICULAR PURPOSE.  See the
#  GNU General Public License for more details.
#
#  You should have received a copy of the GNU General Public License
#  along with this program; if not, write to the Free Software
#  Foundation, Inc., 59 Temple Place - Suite 330, Boston, MA  02111-1307, USA.
#
#  Revision:
#       $Id$
#
#

use File::chdir;
use HTML::Entities;


print "\n\nLedger-SMB login: ";
my $login = <STDIN>;
chomp($login);


print "\nLedger-SMB password: ";
system("stty -echo");
my $pwd = <STDIN>;
system("stty echo");
chomp($pwd);
print "\n\n";

$cmd = "login=" . $login . '&password=' . $pwd . '&path=bin&action=login';

$signin = runLScmd("./login.pl",$cmd);

if ( $signin =~ m/Error:/ ) {

	print "\nLogin error\n";
	exit;

}


while (<main::DATA>) {

	chomp;
	@rec = split(/\|/);

	$arg = 'path=bin/mozilla&login=' . $login . '&password=' . $pwd .
		'&action='       . escape(substr($rec[0],0,35)) .
		'&db='           . $rec[1] .
		'&name='         . escape(substr($rec[2],0,35)) .
		'&vendornumber=' . $rec[3] .
		'&address1='     . escape(substr($rec[4],0,35)) .
		'&address2='     . escape(substr($rec[5],0,35)) .
		'&city='         . escape(substr($rec[6],0,35)) .
		'&state='        . escape(substr($rec[7],0,35)) .
		'&zipcode='      . escape(substr($rec[8],0,35)) .
		'&country='      . escape(substr($rec[9],0,35)) .
		'&phone='        . escape(substr($rec[10],0,20)) .
		'&tax_2150=1' .
		'&taxaccounts=2150' .
		'&taxincluded=0' .
		'&terms=0';

	$rc=runLScmd("./ct.pl",$arg);

	if ($rc =~ m/Vendor saved!/) {

		print "$rec[2] SAVED\n";

	} else {

		print "$rec[2] ERROR\n";

	}

}


$cmd = "login=" . $login . '&password=' . $pwd . '&path=bin&action=logout';

$signin = runLScmd("./login.pl",$cmd);

if ( $signin =~ m/Error:/ ) {

    print "\nLogout error\n";

}

exit;


#*******************************************************
# Subroutines
#*******************************************************


sub runLScmd {

    my $cmd  = shift;
    my $args = shift;
    my $i    = 0;
    my $results;

    local $CWD = "/usr/local/ledger-smb/";

    $cmd = $cmd . " \"" . $args . "\"";

    $results = `$cmd 2>&1`;

    return $results;

}

sub escape {

    my $str = shift;

    if ($str) {

	decode_entities($str);
	$str =~ s/([^a-zA-Z0-9_.-])/sprintf("%%%02x", ord($1))/ge;
    }

    return $str;

}


#*******************************************************
# Record Format
#*******************************************************
#
# action | db | name | vendornumber | address1 | address2 | city | state | zipcode | country | phone
#

__END__
save|vendor|Parts are Us|1377|238 Riverview|Suite 11|Cheese Head|WI|56743|USA|555-123-3322|
save|vendor|Widget Heaven|1378|41 S. Riparian Way||Show Me|MO|39793|USA|555-231-3309|
save|vendor|Consolidated Spackle|1379|1010 Binary Lane|Dept 1101|Beverly Hills|CA|90210|USA|555-330-7639 x772|

 	  	 
\end{verbatim}


\clearpage 


\part{Technical Overview}


\section{Basic Architecture}

Ledger-SMB is a web-based Perl program that interfaces with PostgreSQL
using the relevant Perl modules. The code is well partitioned, and
the main operation modules are written in an object oriented way.


\subsection{The Software Stack}

%
\begin{figure}[hbtp]
 \label{fig-sl-stack} \setlength{\unitlength}{3947sp}%
%
\begingroup\makeatletter\ifx\SetFigFont\undefined%
\gdef\SetFigFont#1#2#3#4#5{%
  \reset@font\fontsize{#1}{#2pt}%
  \fontfamily{#3}\fontseries{#4}\fontshape{#5}%
  \selectfont}%
\fi\endgroup%
\begin{picture}(4824,4824)(3589,-5173)
\thinlines
{\color[rgb]{0,0,0}\put(3601,-1261){\framebox(4800,900){}}
}%
{\color[rgb]{0,0,0}\put(6001,-3961){\framebox(2400,1200){}}
}%
{\color[rgb]{0,0,0}\put(3601,-5161){\framebox(4800,1200){}}
}%
{\color[rgb]{0,0,0}\put(3601,-2761){\framebox(4800,1500){}}
}%
{\color[rgb]{0,0,0}\put(3601,-3961){\framebox(2400,1200){}}
}%
\put(7201,-3361){\makebox(0,0)[b]{\smash{\SetFigFont{17}{20.4}{\rmdefault}{\mddefault}{\updefault}{\color[rgb]{0,0,0}PostgreSQL}%
}}}
\put(6001,-4561){\makebox(0,0)[b]{\smash{\SetFigFont{17}{20.4}{\rmdefault}{\mddefault}{\updefault}{\color[rgb]{0,0,0}Operating System}%
}}}
\put(6001,-811){\makebox(0,0)[b]{\smash{\SetFigFont{17}{20.4}{\rmdefault}{\mddefault}{\updefault}{\color[rgb]{0,0,0}LedgerSMB}%
}}}
\put(4201,-3361){\makebox(0,0)[lb]{\smash{\SetFigFont{17}{20.4}{\rmdefault}{\mddefault}{\updefault}{\color[rgb]{0,0,0}Apache}%
}}}
\put(6001,-1936){\makebox(0,0)[b]{\smash{\SetFigFont{17}{20.4}{\rmdefault}{\mddefault}{\updefault}{\color[rgb]{0,0,0}Perl}%
}}}
\end{picture}
 


\caption{The Ledger-SMB software stack in a Typical Implementation}
\end{figure}


Ledger-SMB runs in a Perl interpreter. I do not currently know if
it is possible to run it with Perl2C or other language converters
to run in other environments. However, except for high-capacity environments,
Perl is a good language choice for this sort of program.

Ledger-SMB used to support DB2 and Oracle as well as PostgreSQL. However,
currently some of the functionality is implemented using PostgreSQL
user-defined functions. These would need to be ported to other database
managers in order to make the software work on these. It should not
be too hard, but the fact that it has not been done yet may mean that
there is no real demand for running the software under other RDBMS's.

One can substitute other web servers for Apache. Normally Ledger-SMB
is run as a CGI program but it may be possible to run it in the web
server process (note that this may not be entirely thread-safe).

The operating system can be any that supports a web server and Perl
(since PostgreSQL need not run on the same system). However, there
are a few issues running Ledger-SMB on Windows (most notably in trying
to get Postscript documents to print properly).

On the client side, any web-browser will work. Currently, the layout
is different for Lynx (which doesn't support frames), and the layout
is not really useful under eLinks (the replacement for Lynx which
does support frames). Some functionality requires Javascript to work
properly, though the application is usable without these features.


\subsection{Capacity Planning}

Some companies may ask how scalable Ledger-SMB is. In general, it
is assumed that few companies are going to have a need for a high-concurrency
accounting system. However, with all the features available in Ledger-SMB,
the staff that may have access to some of the application may be senior
enough to make the question worthwhile.

This question also becomes more important when companies might look
at integrating Ledger-SMB with a CRM solution, online store, or other
environment. This section looks at a number of the known issues and
their solutions.


\subsubsection{Scalability Strategies}

Ledger-SMB is a fairly standard web-based application. However,
sometimes the database schema changes during upgrades. In these cases,
it becomes impossible to use different versions of the software against
the same database version safely. Ledger-SMB checks the version of
the database and if the version is higher than the version of the
software that is running, will refuse to run.

Therefore although one strategy might be to run several front-end
web servers with Ledger-SMB, in reality this can be a bit of a problem.
One solution is to take half of the front-end servers off-line while
doing the initial upgrade, and then take the other offline to upgrade
when these are brought back online.

The database manager is less scalable in the sense that one cannot
just add more database servers and expect to carry on as normal. However,
aside from the known issues listed below, there are few performance
issues with it. If complex reports are necessary, these can be moved
to a replica database (perhaps using Slony-I).

If this solution is insufficient for database scalability, one might
be able to move staff who do not need real-time access to new entries
onto a PG-Pool/Slony-I cluster where new transactions are entered
on the master and other data is looked up on the replica. In certain
circumstances, one can also offload a number of other queries from
the master database in order to minimize the load. Ledger-SMB has
very few issues in the scalability of the application.


\subsubsection{Database Maintenance}

PostgreSQL uses a technique called Multi-version Concurrency Control
(MVCC) to provide a snapshot of the database at the beginning of a
statement or transaction (depending on the transaction isolation level).
When a row is updated, PostgreSQL leaves the old row in the database,
and inserts a new version of that row into the table. Over time, unless
those old rows are removed, performance can degrade as PostgreSQL
has to search through all the old versions of the row in order to
determine which one ought to be the current one.

Due to the way the SQL statements are executed in Ledger-SMB, most
inserts will also create a dead row.

A second problem occurs in that each transaction is given a transaction
id. These id's are numbered using 32-bit integers. If the transaction
id wraps around (prior to 8.1), data from transactions that appear
(due to the wraparound) to be in the future suddenly becomes inaccessible.
This problem was corrected in PostgreSQL 8.1, where the database will
refuse to accept new transactions if the transaction ID gets too close
to a wraparound. So while the problem is not as serious in 8.1, the
application merely becomes inaccessible rather than displaying apparent
data loss. Wraparound would occur after about a billion transactions
between all databases running on that instance of PostgreSQL.

Prior to 8.1, the main way to prevent both these problems was to run
a periodic vacuumdb command from cron (UNIX/Linux) or the task scheduler
(Windows). In 8.1 or later, autovacuum capabilities are part of the
back-end and can be configured with the database manager. See the
PostgreSQL documentation for treatment of these subjects.

In general, if performance appears to be slowly degrading, one should
try to run vacuumdb -z from the shell in order to attempt to reclaim
space and provide the planner with accurate information about the
size and composition of the tables. If this fails, then one can go
to other methods of determining the bottleneck and what to do about
it.


\subsubsection{Known issues}

The PostgreSQL planner assumes a minimum page size of ten pages for
a physically empty table. The reasoning behind this choice is that
a table could grow rapidly and one could end up with bad database
performance if the planner assumes a very small table.

However, if you end up with joins between a very large table with
millions of rows and a physically empty table, one can end up with
a very bad query plan. In this case, the planner will choose a nested
loop join and run through this loop for every row in the large table.
As a result, performance will suddenly drop once the large table becomes
too large to effectively do index scans of the join criteria on both
tables. This problem most often occurs when people have no warehouses,
departments, or projects defined and are running systems with a large
number of transactions (such as a point of sale environment).

Last time I saw this problem, the server would wait for thirty seconds
to display a new point of sale screen while the server CPU activity
would spike to 100\%.

One solution is to define one warehouse, department, and project,
and then run vacuumdb -z from the shell to force the planner to acknowledge
these tables as single-row tables. The other option is to go into
the source code and edit the database queries to omit unused tables.


\section{Customization Possibilities}

Ledger-SMB is designed to be customized relatively easily and rapidly.
In general, the source code is well written and compartmentalized.
This section covers the basic possibilities involving customization.


\subsection{Brief Guide to the Source Code}

Ledger-SMB is an application with over 34000 lines of code. While
it is not possible to cover the entire application here, a brief overview
of the source code is in order.

In the root of the install directory, one will find a setup.pl program,
a number of other .pl programs, and a number of directories. The setup.pl
program is used to update or install Ledger-SMB. The other .pl programs
provide a basic set of services for the framework (including authentication)
and then pass the work on to the data entry screen file in the bin
directory.

The bin directory contains another directory for each terminal type.
The main two offered are lynx and mozilla. Lynx would be used for
web browsers that do not support frames and is ideal for a text-mode
VGA terminal. Mozilla is the terminal type used for most other web
browsers. The perl files within these directories provides the user
interface of the software.

The css directory in the root install directory contains CSS documents
to provide various stylesheets one can select for changing various
aspects of the look and feel of the application.

The locale directory contains translation files that Ledger-SMB uses
to translate between different languages. One could add translations
to these files if necessary.

The LSMB directory is where the Perl modules reside that provide the
core business logic in Ledger-SMB. These modules provide functionality
such as form handling, email capabilities, and access to the database
through its at least partially object oriented API.

Finally, the sql directory provides the database schemas and upgrade
scripts.


\subsection{Data Entry Screens}

One can customize the data entry screens to optimize work flow, display
additional information, etc.


\subsubsection{Examples}

We set up hot keys for payment lines, automatically focused the keyboard
on the last partnumber field, removed separate print and post buttons
to ensure that invoices were always printed and posted together, and
removed the ability to print to the screen, and even the ability to
scan items in when an invoice was received (using a portable data
terminal) and import this data into Ledger-SMB. Finally we added the
ability to reconcile the till online in a paperless manner.

For another customer, we added the ability to print AR invoices in
plain text format and added templates (based on the POS sales template)
to do this.


\subsection{Extensions}

One can add functionality to the Perl modules in the LSMB directory
and often add missing functions easily.


\subsubsection{Examples}

For one customer, we added a module to take data from a portable data
terminal collected when inventory items were taken and use that to
add shrinkage and loss adjustments. We also extended the parts model
to add a check id flag (for alcohol sales) and added this flag to
the user interface.

For another customer, we added a complex invoice/packing slip tracking
system that allowed one to track all the printed documents associated
with an order or invoice.


\subsection{Templates}

As noted before templates can be modified or extended, though sometimes
this involves extending the user interface scripts. Most templates
are easy enough to modify. 


\subsubsection{Examples}

For one customer we added text-only invoices for AR and AP transactions/Invoices
and an ability to use Javascript in them to automatically print them
on load.


\subsection{Reports}

The fact that all the data is available within the database manager
is a huge advantage of Ledger-SMB over Quickbooks and the like. The
rapid development of reports allows for one to easily develop reports
of any sort within Ledger-SMB.


\subsubsection{Examples}

For one customer, we developed a report of parts sold and received
during arbitrary time frames. The report allows one to go back and
look up the invoices involved.


\section{Integration Possibilities}

An open database system and programming API allows for many types
of integration. There are some challenges, but in the end, one can
integrate a large number of tools.


\subsection{Reporting Tools}

Any reporting tool which can access the PostgreSQL database can be
used with Ledger-SMB for custom reporting. These can include programs
like Microsoft Access and Excel (using the ODBC drivers), PgAccess
(A PostgreSQL front-end written in TCL/Tk with a similar feel to Access),
Rekall, Crystal Reports, OpenOffice and more.


\subsubsection{Examples}

We have created spreadsheets of the summaries of activity by day and
used the ODBC driver to import these into Excel. Excel can also read
HTML tables, so one can use PostgreSQL to create an HTML table of
the result and save it with a .xls extension so that Windows opens
it with Excel. These could then be served via the same web server
that serves Ledger-SMB.


\subsection{Line of Business Tools on PostgreSQL}

Various line of business tools have been written using PostgreSQL
in large part due to the fact that it is far more mature than MySQL
in areas relating to data integrity enforcement, transactional processing,
and the like. These tools can be integrated with Ledger-SMB in various
ways. One could integrate this program with the HERMES CRM framework,
for example.


\subsubsection{Known Issues}

Ledger-SMB uses a single 'id' sequence across many tables. At the
same time it is expected that these tables do not have identical id
values in their records as they are used as a sort of pseudo-foreign
key by the acc\_trans table which stores the financial transaction
information.

If the integration solution does not keep this in mind, it is possible
to create a situation where the account transactions are ambiguously
associated with a number of different types of financial transactions.
This would lead to a large number of problems.


\subsubsection{Strategies}

In general, it is advisable to run all such programs that benefit
from integration in the same database but under different schemas.
This allows PostgreSQL to become the main method of synchronizing
the data in real time. However, sometimes this can require dumping
the database, recreating the tables etc. in a different schema and
importing the data back into Ledger-SMB.

One possibility for this sort of integration is to use database triggers
to replicate the data between the applications in real-time. This
can avoid the main issue of duplicate id's. One issue that can occur
however relates to updates. If one updates a customer record in HERMES,
for example, how do we know which record to update in Ledger-SMB?
There are solutions to this problem but they do require some forethought.

A second possibility is to use views to allow one application to present
the data from the other as its own. This can be cleaner regarding
update issues, but it can also pose issues regarding duplicate id
fields.


\subsubsection{Examples}

Others have integrated L'ane POS and Ledger-SMB in order to make it
work better with touch screen devices. Still others have successfully
integrated Ledger-SMB and Interchange. In both cases, I believe that
triggers were used to perform the actual integration.


\subsection{Line of Business Tools on other RDBMS's}

Often there are requests to integrate Ledger-SMB with applications
like SugarCRM, OSCommerce, and other applications running on MySQL
or other database managers. This is a far more complex field and it
requires a great deal more effort than integrating applications within
the same database.


\subsubsection{Strategies}

Real-time integration is not always possible. MySQL does
not support the SQL extension SQL/MED (Management of External Data) that
supports non-MySQL data sources so it is not possible to replicate the 
data in real-time. Therefore
one generally resorts to integrating the system using time-based updates.
Replication may be somewhat error-prone unless the database manager
supports triggers (first added to MySQL in 5.0) or other mechanisms
to ensure that all changed records can be detected and replicated.
In general, it is usually advisable to add two fields to the record--
one that shows the insert time and one that shows the last update.

Additionally, I would suggest adding additional information to the
Ledger-SMB tables so that you can track the source record from the
other application in the case of an update.

In general, one must write replication scripts that dump the information
from one and add it to the other. This must go both ways.


\subsubsection{Integration Products and Open Source Projects}

While many people write Perl scripts to accomplish the replication,
an open source project exists called DBI-Link. This package requires
PL/Perl to be installed in PostgreSQL, and it allows PostgreSQL to
present any data accessible via Perl's DBI framework as PostgreSQL
tables. DBI-Link can be used to allow PostgreSQL to pull the data
from MySQL or other database managers.

DBI-Link can simplify the replication process by reducing the operation
to a set of SQL queries.


\section{Customization Guide}

This section is meant to provide a programmer with an understanding
of the technologies enough information to get up to speed quickly
and minimize the time spent familiarizing themselves with the software.
Topics in this section are listed in order of complexity. As it appeals
to a narrower audience than previous discussions of this topic, it
is listed separately.


\subsection{General Information}

The main framework scripts (the ar.pl, ap.pl, etc. scripts found in
the root of the installation directory) handle such basic features
as instantiating the form object, ensuring that the user is logged
in, and the like. They then pass the execution off to the user interface
script (usually in the bin/mozilla directory).

Ledger-SMB in many ways may look sort of object oriented in its design,
but in reality, it is far more data-driven than object oriented. The
Form object is used largely as a global symbol table and also as a
collection of fundamental routines for things like database access.
It also breaks down the query string into sets of variables which
are stored in its attribute hash table.

In essence one can and often will store all sorts of data structures
in the primary Form object. These can include almost anything. It
is not uncommon to see lists of hashes stored as attributes to a Form
object.


\subsection{Customizing Templates}

Templates are used to generate printed checks, invoices, receipts,
and more in Ledger-SMB. Often the format of these items does not fit
a specific set of requirements and needs to be changed. This document
will not include \LaTeX{} or HTML instruction, but will include a
general introduction to editing templates. Also, this is not intended
to function as a complete reference.

Template instructions are contained in tags \textless?lsmb and ?\textgreater.
The actual parsing is done by the parse\_template function in LSMB/Form.pm.


\subsubsection{Page Breaks in \LaTeX{}}

The first tag one will see with \LaTeX{}\ templates is \textless?lsmb pagebreak
num1 num2 num3 ?\textgreater

\begin{itemize}
\item num1 represents characters per line 
\item num2 represents lines on first page 
\item num3 represents lines on second page. 
\end{itemize}
The pagebreak block is terminated by \textless?lsmb end pagebreak ?\textgreater.
Any text within the pagebreak block is ignored by the template.


\subsubsection{Conditionals}

\begin{itemize}
\item \textless?lsmb if not varname ?\textgreater tells the parser to
include the next block only if varname was posted by the submitting
form (or set via the form hash elsewhere in the scripts). The block
ends with \textless?lsmb end varname ?\textgreater 
\item \textless?lsmb if varname ?\textgreater tells the parser to include the
block if varname was posted in the submitting form (or set via
the form hash elsewhere in the scripts). The block ends with \textless?lsmb end
varname ?\textgreater 
\item Lines conditionals are otherwise ignored by the parser. 
\item Conditionals cannot be nested, but IF's can be nested inside loops.
\end{itemize}

\subsubsection{Loops}

\textless?lsmb foreach varname ?\textgreater is used to iterate through
a list of vars set by the user interface system (usually one of the
files under bin/mozilla (or otherwise). The block is repeated for
each varname in a list. Block ends with \textless?lsmb end varname ?\textgreater


\subsubsection{File Inclusion}

\begin{itemize}
\item Files may be included with the syntax \textless?lsmb include template\_name
?\textgreater
where templatename is the name of the template within the current
template directory (usually templates/\$username/).  Note that for \LaTeX\ 
templates, the input or include functionalities might be better suited for many
many things.
\item Cannot be used with conditionals 
\item Filenames cannot use slashes (/) or .. due to directory transversal
considerations. 
\item Files can force other files to be included, but the same file cannot
be included more than once. 
\end{itemize}

\subsubsection{Cross-referencing and multiple passes of \LaTeX{}}

In \LaTeX{}\ cross-references require two passes with latex to resolve.
This is because the type is set page by page and the program really
doesn't know on which page a given reference will fall. This becomes
an even larger issue where floats are concerned as they can move between
pages for formatting reasons.

In rare cases, cross-references may point at incorrect pages even
with two passes (if the inclusion of the cross-reference data moves
the object to another page). In this case you will need to use three
passes of \LaTeX{}\ in order to have accurate references.

Ledger-SMB as of the time of this writing (2.6.8) only makes one pass
at the \LaTeX{}\ file. To force it to make more than one pass, open
Form.pm with your favorite text editor. Look for the line:

system(\char`\"{}latex --interaction=nonstopmode \$self-\textgreater$\lbrace$
tmpfile$\rbrace$\
\textgreater\ \$ self-\textgreater$\lbrace$ tmpfile$\rbrace$
.err\char`\"{});

Duplicate this line for two passes, or add two copies if you need
three passes.


\subsubsection{Variable Substitution}

The following format is used for variable substitution: 

\begin{itemize}
\item \textless?lsmb varname options ?\textgreater Options are one or more
(whitespace separated) of: 

\begin{itemize}
\item align=left/right/center 
\item width=chars where chars is the width in characters before wrapping 
\item offset=chars where chars is the number of spaces to (depending on
alignment). 
\end{itemize}
\end{itemize}

\subsection{Customizing Forms}

Data entry forms and other user interface pieces are in the bin directory.
In Ledger-SMB 1.0.0 and later, symlinks are not generally used.

Each module is identified with a two letter combination: ar, ap, cp,
etc. These combinations are generally explained in the comment headers
on each file.

Execution in these files begins with the function designated by the
form->$\lbrace$action$\rbrace$ variable. This variable is usually
derived from configuration parameters in the menu.ini or the name
of the button that was clicked on to submit the previous page. Due
to localization requirements, the following process is used to determine
the appropriate action taken:

The \$locale-\textgreater getsub routine is called. This routine
checks the locale package to determine if the value needs to be translated
back into an appropriate LSMB function. If not, the variable is lower-cased,
and all spaces are converted into underscores.

In general there is no substitute for reading the code to understand
how this can be customized and how one might go about doing this.


\subsection{Customizing Modules}

The Perl Modules (.pm files) in the LedgerSMB directory contain the
main business logic of the application including all database access.
Most of the modules are relatively easy to follow, but the code quality 
leaves something to be desired.  The API calls themselves are 
likely to be replaced in the future, so work on documenting API calls is
now focused solely on those calls that are considered to be stable.
At the moment, the best place to request information on the API's is on
the Developmers' Email List (\mailto{ledger-smb-devel@lists.sourceforge.net}).

Many of these modules have a fair bit of dormant code in them which
was written for forthcoming features, such as payroll and bills of
materials.

One can add a new module through the normal means and connect it to
other existing modules.


\subsubsection{Database Access}

The \$form object provides two methods for accessing the database.
The \$form-\textgreater dbconnect(\%myconfig) method commits each
individual statement as its own transaction. The \$form-\textgreater
dbconnect\_noauto(\%myconfig) method requires a manual commit. Both
these functions are thin wrappers around the standard Perl DBI operations.


\subsection{CLI Examples}
Louis Moore contributed some SQL-Ledger CLI examples that still work for 
LedgerSMB.  You can find his page at \url{http://www.sql-ledger.org/cgi-bin/nav.pl?page=contrib/moore.html&title=Louis\%20B.\%20Moore}.

\clearpage 


\part{Appendix}

\appendix
%dummy comment inserted by tex2lyx to ensure that this paragraph is not empty



\section{Where to Go for More Information}

There are a couple of relevant sources of information on Ledger-SMB
in particular.

The most important resources are the LedgerSMB web site 
(\url{http://www.ledgersmb.org}) and the email lists found at our Sourceforge
site.

In addition, it is generally recommended that the main bookkeeper
of a company using Ledger-SMB work through at least one accounting
textbook. Which textbook is not as important as the fact that a textbook
is used.


\section{Quick Tips}


\subsection{Understanding Shipping Addresses and Carriers}

Each customer can have a default shipping address. This address is
displayed prominantly in the add new customer screen. To change the
shipping address for a single order, one can use the ship to button
at the bottom of the quote, order, or invoice screen.

The carrier can be noted in the Ship Via field. However, this is a
freeform field and is largely used as commentary (or instructions
for the shipping crew).


\subsection{Handling bad debts}

In the event that a customer's check bounces or a collection requirement
is made, one can flag the customer's account by setting the credit
limit to a negative number.

If a debt needs to be written off, one can either use the allowance
method (by writing it against the contra asset account of \char`\"{}Allowance
for Bad Debts\char`\"{} or using the direct writeoff method where
it is posted as an expense.


\section{Step by Steps for Vertical Markets}


\subsection{Common Installation Errors}

\begin{itemize}
\item Ledger-SMB is generally best installed in its own directory outside
of the wwwroot directory. While it is possible to install it inside
the wwwroot directory, the instructions and the faq don't cover the
common problems here. 
\item When the chart of accounts (COA) is altered such that it is no longer
set up with appropriate items, you can make it impossible to define
goods and services properly. In general, until you are familiar with
the software, it is best to rename and add accounts rather than deleting
them. 
\end{itemize}

\subsection{Retail With Light Manufacturing}

For purposes of this example we will use a business that assembles
computers and sells them on a retail store. 

\begin{enumerate}
\item Install Ledger-SMB.
\item Set preferences, and customize chart of accounts.

\begin{enumerate}
\item Before customizing the COA it is suggested that you consult an accountant. 
\end{enumerate}
\item Define Goods, Labor, and Services as raw parts ordered from the vendors. 

\begin{itemize}
\item These are located under the goods and services menu node. 
\end{itemize}
\item Define assemblies.

\begin{itemize}
\item These are also located under goods and services. 
\item Component goods and services must be defined prior to creating assembly.
\end{itemize}
\item Enter an AP Invoice to populate inventory with proper raw materials. 

\begin{itemize}
\item One must generally add a generic vendor first. The vendor is added
under AP-\textgreater Vendors-\textgreater Add Vendor. 
\end{itemize}
\item To pay an AP invoice like a check, go to cash-$>$payment. Fill out 
approrpiate fields and click print. 

\begin{itemize}
\item Note that one should select an invoice and enter in the payment amount
in the appropriate line of the invoice list. If you add amounts to
the master amount list, you will find that they are added to the amount
paid on the invoice as a prepayment. 
\item The source field is the check number. 
\end{itemize}
\item Stock assemblies 
\item One can use AR Invoices or the POS interface to sell goods and services. 

\begin{itemize}
\item Sales Invoice 

\begin{itemize}
\item Can be generated from orders or quotations 
\item Cannot include labor/overhead except as part of an assembly 
\item One can make the mistake of printing the invoice and forgetting to
post it. In this event, the invoice does not officially exist in the
accounting system. 
\item For new customers, you must add the customer first (under AR-\textgreater
Customers-\textgreater Add Customer. 
\end{itemize}
\item POS Interface 

\begin{itemize}
\item Cannot include labor/overhead except as part of an assembly 
\item Printing without posting is often even easier in the POS because of
the rapid workflow. Yet it is just as severe a problem. 
\end{itemize}
\item Ecommerce and Mail Order Operations 

\begin{itemize}
\item See the shipping workflow documentation above. 
\end{itemize}
\item Customers are set up by going to AR-\textgreater Customers-\textgreater
Add Customer (or the equivalent localized translation). The appropriate
fields are filled out and the buttons are used at the bottom to save
the record and optionally use it to create an invoice, etc. 

\begin{itemize}
\item Saving a customer returns to the customer screen. After the appropriate
invoice, transaction, etc. is entered and posted, Ledger-SMB will
return to the add customer screen. 
\end{itemize}
\end{itemize}
\item One can use the requirements report to help determine what parts need
to be ordered though one cannot generate PO's directly from this report.
\end{enumerate}
Note, the needs of Ledger-SMB are mostly useful for light manufacturing
operations (assembling computers, for example). More manufacturing
capabilities are expected to be released in the next version.

A custom assembly is a bit difficult to make. One must add the assembly
prior to invoice (this is not true of goods and services). If the
assembly is based on a different assembly but may cost more (due to
non-standard parts) you can load the old assembly using Goods and
Services-\textgreater Reports-\textgreater Assemblies and then make
necessary changes (including to the SKU/Partnumber) and save it as
new.

Then one can add it to the invoice.


\section{Glossary}

\begin{description}
\item [{BIC}] Bank Identifier Code is often the same as the S.W.I.F.T.
code. This is a code for the bank a customer uses for automated money
transfers. 
\item [{COGS}] is Cost of Goods Sold. When an item is sold, then the expense
of its purchase is accrued as attached to the income of the sale.
It is tracked as COGS. 
\item [{Credit}] : A logical transactional unit in double entry accounting.
It is the opposite of a debit. Credits affect different account types
as follows: 

\begin{description}
\item [{Equity}] : Credits are added to the account when money is invested
in the business. 
\item [{Asset}] : Credits are added when money is deducted from an asset
account. 
\item [{Liability}] : Credits are added when money is owed to the business
account. 
\item [{Income}] : Credits are added when income is earned. 
\item [{Expense}] : Credits are used to apply adjustments at the end of
accounting periods to indicate that not all the expense for an AP
transaction has been fully accrued. 
\end{description}
\item [{Debit}] : A logical transactional unit in double entry accounting
systems. It is the opposite of a credit. Debits affect different account
types as follows: 

\begin{description}
\item [{Equity}] : Debits are added when money is paid to business owners. 
\item [{Asset}] : Debits are added when money is added to an account. 
\item [{Liability}] : Debits are added when money that is owed is paid
off. 
\item [{Income}] : Debits are used to temporarily adjust income to defer
unearned income to the next accounting period. 
\item [{Expense}] : Debits are added as expenses are incurred. 
\end{description}
\item [{IBAN}] International Bank Account Number is related to the BIC
and is used for cross-border automated money transfers. 
\item [{List}] Price is the recommended retail price. 
\item [{Markup}] is the percentage increase that is applied to the last
cost to get the sell price. 
\item [{ROP}] Re-order point. Items with fewer in stock than this will
show up on short reports. 
\item [{Sell}] Price is the price at which the item is sold. 
\item [{SKU}] Stock Keeping Unit: a number designating a specific product.
\item [{Source}] Document : a paper document that can be used as evidence
that a transaction occurred. Source documents can include canceled
checks, receipts, credit card statements and the like. 
\item [{Terms}] is the number of days one has to pay the invoice. Most
businesses abbreviate the terms as Net n where n is the number of
days. For example, Net 30 means the customer has 30 days to pay the
net due on an invoice before it is late and incurs late fees.  Sometimes
you will see 2 10 net 30 or similar.  This means 2\% discount if paid within
10 days but due within 30 days in any case.
\end{description}
\section{GNU Free Documentation License}
\label{fdl}
%\label{label_fdl}

 \begin{center}

       Version 1.2, November 2002


 Copyright \copyright 2000,2001,2002  Free Software Foundation, Inc.
 
 \bigskip
 
     51 Franklin St, Fifth Floor, Boston, MA  02110-1301  USA
  
 \bigskip
 
 Everyone is permitted to copy and distribute verbatim copies
 of this license document, but changing it is not allowed.
\end{center}


\begin{center}
{\bf\large Preamble}
\end{center}

The purpose of this License is to make a manual, textbook, or other
functional and useful document "free" in the sense of freedom: to
assure everyone the effective freedom to copy and redistribute it,
with or without modifying it, either commercially or noncommercially.
Secondarily, this License preserves for the author and publisher a way
to get credit for their work, while not being considered responsible
for modifications made by others.

This License is a kind of "copyleft", which means that derivative
works of the document must themselves be free in the same sense.  It
complements the GNU General Public License, which is a copyleft
license designed for free software.

We have designed this License in order to use it for manuals for free
software, because free software needs free documentation: a free
program should come with manuals providing the same freedoms that the
software does.  But this License is not limited to software manuals;
it can be used for any textual work, regardless of subject matter or
whether it is published as a printed book.  We recommend this License
principally for works whose purpose is instruction or reference.


\begin{center}
{\Large\bf 1. APPLICABILITY AND DEFINITIONS}
\addcontentsline{toc}{subsection}{1. APPLICABILITY AND DEFINITIONS}
\end{center}

This License applies to any manual or other work, in any medium, that
contains a notice placed by the copyright holder saying it can be
distributed under the terms of this License.  Such a notice grants a
world-wide, royalty-free license, unlimited in duration, to use that
work under the conditions stated herein.  The \textbf{"Document"}, below,
refers to any such manual or work.  Any member of the public is a
licensee, and is addressed as \textbf{"you"}.  You accept the license if you
copy, modify or distribute the work in a way requiring permission
under copyright law.

A \textbf{"Modified Version"} of the Document means any work containing the
Document or a portion of it, either copied verbatim, or with
modifications and/or translated into another language.

A \textbf{"Secondary Section"} is a named appendix or a front-matter section of
the Document that deals exclusively with the relationship of the
publishers or authors of the Document to the Document's overall subject
(or to related matters) and contains nothing that could fall directly
within that overall subject.  (Thus, if the Document is in part a
textbook of mathematics, a Secondary Section may not explain any
mathematics.)  The relationship could be a matter of historical
connection with the subject or with related matters, or of legal,
commercial, philosophical, ethical or political position regarding
them.

The \textbf{"Invariant Sections"} are certain Secondary Sections whose titles
are designated, as being those of Invariant Sections, in the notice
that says that the Document is released under this License.  If a
section does not fit the above definition of Secondary then it is not
allowed to be designated as Invariant.  The Document may contain zero
Invariant Sections.  If the Document does not identify any Invariant
Sections then there are none.

The \textbf{"Cover Texts"} are certain short passages of text that are listed,
as Front-Cover Texts or Back-Cover Texts, in the notice that says that
the Document is released under this License.  A Front-Cover Text may
be at most 5 words, and a Back-Cover Text may be at most 25 words.

A \textbf{"Transparent"} copy of the Document means a machine-readable copy,
represented in a format whose specification is available to the
general public, that is suitable for revising the document
straightforwardly with generic text editors or (for images composed of
pixels) generic paint programs or (for drawings) some widely available
drawing editor, and that is suitable for input to text formatters or
for automatic translation to a variety of formats suitable for input
to text formatters.  A copy made in an otherwise Transparent file
format whose markup, or absence of markup, has been arranged to thwart
or discourage subsequent modification by readers is not Transparent.
An image format is not Transparent if used for any substantial amount
of text.  A copy that is not "Transparent" is called \textbf{"Opaque"}.

Examples of suitable formats for Transparent copies include plain
ASCII without markup, Texinfo input format, LaTeX input format, SGML
or XML using a publicly available DTD, and standard-conforming simple
HTML, PostScript or PDF designed for human modification.  Examples of
transparent image formats include PNG, XCF and JPG.  Opaque formats
include proprietary formats that can be read and edited only by
proprietary word processors, SGML or XML for which the DTD and/or
processing tools are not generally available, and the
machine-generated HTML, PostScript or PDF produced by some word
processors for output purposes only.

The \textbf{"Title Page"} means, for a printed book, the title page itself,
plus such following pages as are needed to hold, legibly, the material
this License requires to appear in the title page.  For works in
formats which do not have any title page as such, "Title Page" means
the text near the most prominent appearance of the work's title,
preceding the beginning of the body of the text.

A section \textbf{"Entitled XYZ"} means a named subunit of the Document whose
title either is precisely XYZ or contains XYZ in parentheses following
text that translates XYZ in another language.  (Here XYZ stands for a
specific section name mentioned below, such as \textbf{"Acknowledgements"},
\textbf{"Dedications"}, \textbf{"Endorsements"}, or \textbf{"History"}.)  
To \textbf{"Preserve the Title"}
of such a section when you modify the Document means that it remains a
section "Entitled XYZ" according to this definition.

The Document may include Warranty Disclaimers next to the notice which
states that this License applies to the Document.  These Warranty
Disclaimers are considered to be included by reference in this
License, but only as regards disclaiming warranties: any other
implication that these Warranty Disclaimers may have is void and has
no effect on the meaning of this License.


\begin{center}
{\Large\bf 2. VERBATIM COPYING}
\addcontentsline{toc}{subsection}{2. VERBATIM COPYING}
\end{center}

You may copy and distribute the Document in any medium, either
commercially or noncommercially, provided that this License, the
copyright notices, and the license notice saying this License applies
to the Document are reproduced in all copies, and that you add no other
conditions whatsoever to those of this License.  You may not use
technical measures to obstruct or control the reading or further
copying of the copies you make or distribute.  However, you may accept
compensation in exchange for copies.  If you distribute a large enough
number of copies you must also follow the conditions in section 3.

You may also lend copies, under the same conditions stated above, and
you may publicly display copies.


\begin{center}
{\Large\bf 3. COPYING IN QUANTITY}
\addcontentsline{toc}{subsection}{3. COPYING IN QUANTITY}
\end{center}


If you publish printed copies (or copies in media that commonly have
printed covers) of the Document, numbering more than 100, and the
Document's license notice requires Cover Texts, you must enclose the
copies in covers that carry, clearly and legibly, all these Cover
Texts: Front-Cover Texts on the front cover, and Back-Cover Texts on
the back cover.  Both covers must also clearly and legibly identify
you as the publisher of these copies.  The front cover must present
the full title with all words of the title equally prominent and
visible.  You may add other material on the covers in addition.
Copying with changes limited to the covers, as long as they preserve
the title of the Document and satisfy these conditions, can be treated
as verbatim copying in other respects.

If the required texts for either cover are too voluminous to fit
legibly, you should put the first ones listed (as many as fit
reasonably) on the actual cover, and continue the rest onto adjacent
pages.

If you publish or distribute Opaque copies of the Document numbering
more than 100, you must either include a machine-readable Transparent
copy along with each Opaque copy, or state in or with each Opaque copy
a computer-network location from which the general network-using
public has access to download using public-standard network protocols
a complete Transparent copy of the Document, free of added material.
If you use the latter option, you must take reasonably prudent steps,
when you begin distribution of Opaque copies in quantity, to ensure
that this Transparent copy will remain thus accessible at the stated
location until at least one year after the last time you distribute an
Opaque copy (directly or through your agents or retailers) of that
edition to the public.

It is requested, but not required, that you contact the authors of the
Document well before redistributing any large number of copies, to give
them a chance to provide you with an updated version of the Document.


\begin{center}
{\Large\bf 4. MODIFICATIONS}
\addcontentsline{toc}{subsection}{4. MODIFICATIONS}
\end{center}

You may copy and distribute a Modified Version of the Document under
the conditions of sections 2 and 3 above, provided that you release
the Modified Version under precisely this License, with the Modified
Version filling the role of the Document, thus licensing distribution
and modification of the Modified Version to whoever possesses a copy
of it.  In addition, you must do these things in the Modified Version:

\begin{itemize}
\item[A.] 
   Use in the Title Page (and on the covers, if any) a title distinct
   from that of the Document, and from those of previous versions
   (which should, if there were any, be listed in the History section
   of the Document).  You may use the same title as a previous version
   if the original publisher of that version gives permission.
   
\item[B.]
   List on the Title Page, as authors, one or more persons or entities
   responsible for authorship of the modifications in the Modified
   Version, together with at least five of the principal authors of the
   Document (all of its principal authors, if it has fewer than five),
   unless they release you from this requirement.
   
\item[C.]
   State on the Title page the name of the publisher of the
   Modified Version, as the publisher.
   
\item[D.]
   Preserve all the copyright notices of the Document.
   
\item[E.]
   Add an appropriate copyright notice for your modifications
   adjacent to the other copyright notices.
   
\item[F.]
   Include, immediately after the copyright notices, a license notice
   giving the public permission to use the Modified Version under the
   terms of this License, in the form shown in the Addendum below.
   
\item[G.]
   Preserve in that license notice the full lists of Invariant Sections
   and required Cover Texts given in the Document's license notice.
   
\item[H.]
   Include an unaltered copy of this License.
   
\item[I.]
   Preserve the section Entitled "History", Preserve its Title, and add
   to it an item stating at least the title, year, new authors, and
   publisher of the Modified Version as given on the Title Page.  If
   there is no section Entitled "History" in the Document, create one
   stating the title, year, authors, and publisher of the Document as
   given on its Title Page, then add an item describing the Modified
   Version as stated in the previous sentence.
   
\item[J.]
   Preserve the network location, if any, given in the Document for
   public access to a Transparent copy of the Document, and likewise
   the network locations given in the Document for previous versions
   it was based on.  These may be placed in the "History" section.
   You may omit a network location for a work that was published at
   least four years before the Document itself, or if the original
   publisher of the version it refers to gives permission.
   
\item[K.]
   For any section Entitled "Acknowledgements" or "Dedications",
   Preserve the Title of the section, and preserve in the section all
   the substance and tone of each of the contributor acknowledgements
   and/or dedications given therein.
   
\item[L.]
   Preserve all the Invariant Sections of the Document,
   unaltered in their text and in their titles.  Section numbers
   or the equivalent are not considered part of the section titles.
   
\item[M.]
   Delete any section Entitled "Endorsements".  Such a section
   may not be included in the Modified Version.
   
\item[N.]
   Do not retitle any existing section to be Entitled "Endorsements"
   or to conflict in title with any Invariant Section.
   
\item[O.]
   Preserve any Warranty Disclaimers.
\end{itemize}

If the Modified Version includes new front-matter sections or
appendices that qualify as Secondary Sections and contain no material
copied from the Document, you may at your option designate some or all
of these sections as invariant.  To do this, add their titles to the
list of Invariant Sections in the Modified Version's license notice.
These titles must be distinct from any other section titles.

You may add a section Entitled "Endorsements", provided it contains
nothing but endorsements of your Modified Version by various
parties--for example, statements of peer review or that the text has
been approved by an organization as the authoritative definition of a
standard.

You may add a passage of up to five words as a Front-Cover Text, and a
passage of up to 25 words as a Back-Cover Text, to the end of the list
of Cover Texts in the Modified Version.  Only one passage of
Front-Cover Text and one of Back-Cover Text may be added by (or
through arrangements made by) any one entity.  If the Document already
includes a cover text for the same cover, previously added by you or
by arrangement made by the same entity you are acting on behalf of,
you may not add another; but you may replace the old one, on explicit
permission from the previous publisher that added the old one.

The author(s) and publisher(s) of the Document do not by this License
give permission to use their names for publicity for or to assert or
imply endorsement of any Modified Version.


\begin{center}
{\Large\bf 5. COMBINING DOCUMENTS}
\addcontentsline{toc}{subsection}{5. COMBINING DOCUMENTS}
\end{center}


You may combine the Document with other documents released under this
License, under the terms defined in section 4 above for modified
versions, provided that you include in the combination all of the
Invariant Sections of all of the original documents, unmodified, and
list them all as Invariant Sections of your combined work in its
license notice, and that you preserve all their Warranty Disclaimers.

The combined work need only contain one copy of this License, and
multiple identical Invariant Sections may be replaced with a single
copy.  If there are multiple Invariant Sections with the same name but
different contents, make the title of each such section unique by
adding at the end of it, in parentheses, the name of the original
author or publisher of that section if known, or else a unique number.
Make the same adjustment to the section titles in the list of
Invariant Sections in the license notice of the combined work.

In the combination, you must combine any sections Entitled "History"
in the various original documents, forming one section Entitled
"History"; likewise combine any sections Entitled "Acknowledgements",
and any sections Entitled "Dedications".  You must delete all sections
Entitled "Endorsements".

\begin{center}
{\Large\bf 6. COLLECTIONS OF DOCUMENTS}
\addcontentsline{toc}{subsection}{6. COLLECTIONS OF DOCUMENTS}
\end{center}

You may make a collection consisting of the Document and other documents
released under this License, and replace the individual copies of this
License in the various documents with a single copy that is included in
the collection, provided that you follow the rules of this License for
verbatim copying of each of the documents in all other respects.

You may extract a single document from such a collection, and distribute
it individually under this License, provided you insert a copy of this
License into the extracted document, and follow this License in all
other respects regarding verbatim copying of that document.


\begin{center}
{\Large\bf 7. AGGREGATION WITH INDEPENDENT WORKS}
\addcontentsline{toc}{subsection}{7. AGGREGATION WITH INDEPENDENT WORKS}
\end{center}


A compilation of the Document or its derivatives with other separate
and independent documents or works, in or on a volume of a storage or
distribution medium, is called an "aggregate" if the copyright
resulting from the compilation is not used to limit the legal rights
of the compilation's users beyond what the individual works permit.
When the Document is included in an aggregate, this License does not
apply to the other works in the aggregate which are not themselves
derivative works of the Document.

If the Cover Text requirement of section 3 is applicable to these
copies of the Document, then if the Document is less than one half of
the entire aggregate, the Document's Cover Texts may be placed on
covers that bracket the Document within the aggregate, or the
electronic equivalent of covers if the Document is in electronic form.
Otherwise they must appear on printed covers that bracket the whole
aggregate.


\begin{center}
{\Large\bf 8. TRANSLATION}
\addcontentsline{toc}{subsection}{8. TRANSLATION}
\end{center}


Translation is considered a kind of modification, so you may
distribute translations of the Document under the terms of section 4.
Replacing Invariant Sections with translations requires special
permission from their copyright holders, but you may include
translations of some or all Invariant Sections in addition to the
original versions of these Invariant Sections.  You may include a
translation of this License, and all the license notices in the
Document, and any Warranty Disclaimers, provided that you also include
the original English version of this License and the original versions
of those notices and disclaimers.  In case of a disagreement between
the translation and the original version of this License or a notice
or disclaimer, the original version will prevail.

If a section in the Document is Entitled "Acknowledgements",
"Dedications", or "History", the requirement (section 4) to Preserve
its Title (section 1) will typically require changing the actual
title.


\begin{center}
{\Large\bf 9. TERMINATION}
\addcontentsline{toc}{subsection}{9. TERMINATION}
\end{center}


You may not copy, modify, sublicense, or distribute the Document except
as expressly provided for under this License.  Any other attempt to
copy, modify, sublicense or distribute the Document is void, and will
automatically terminate your rights under this License.  However,
parties who have received copies, or rights, from you under this
License will not have their licenses terminated so long as such
parties remain in full compliance.


\begin{center}
{\Large\bf 10. FUTURE REVISIONS OF THIS LICENSE}
\addcontentsline{toc}{subsection}{10. FUTURE REVISIONS OF THIS LICENSE}
\end{center}


The Free Software Foundation may publish new, revised versions
of the GNU Free Documentation License from time to time.  Such new
versions will be similar in spirit to the present version, but may
differ in detail to address new problems or concerns.  See
http://www.gnu.org/copyleft/.

Each version of the License is given a distinguishing version number.
If the Document specifies that a particular numbered version of this
License "or any later version" applies to it, you have the option of
following the terms and conditions either of that specified version or
of any later version that has been published (not as a draft) by the
Free Software Foundation.  If the Document does not specify a version
number of this License, you may choose any version ever published (not
as a draft) by the Free Software Foundation.


\begin{center}
{\Large\bf ADDENDUM: How to use this License for your documents}
\addcontentsline{toc}{subsection}{ADDENDUM: How to use this License for your documents}
\end{center}

To use this License in a document you have written, include a copy of
the License in the document and put the following copyright and
license notices just after the title page:

\bigskip
\begin{quote}
    Copyright \copyright  YEAR  YOUR NAME.
    Permission is granted to copy, distribute and/or modify this document
    under the terms of the GNU Free Documentation License, Version 1.2
    or any later version published by the Free Software Foundation;
    with no Invariant Sections, no Front-Cover Texts, and no Back-Cover Texts.
    A copy of the license is included in the section entitled "GNU
    Free Documentation License".
\end{quote}
\bigskip
    
If you have Invariant Sections, Front-Cover Texts and Back-Cover Texts,
replace the "with...Texts." line with this:

\bigskip
\begin{quote}
    with the Invariant Sections being LIST THEIR TITLES, with the
    Front-Cover Texts being LIST, and with the Back-Cover Texts being LIST.
\end{quote}
\bigskip
    
If you have Invariant Sections without Cover Texts, or some other
combination of the three, merge those two alternatives to suit the
situation.

If your document contains nontrivial examples of program code, we
recommend releasing these examples in parallel under your choice of
free software license, such as the GNU General Public License,
to permit their use in free software.

 
\end{document}
